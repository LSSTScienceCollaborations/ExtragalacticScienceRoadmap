This document is a draft road-map for Black Hole science with LSST.
In composing it, I have followed the organization of the LSST Science
Book, specifically Chapter 10 on AGNs.  That way the authors of the
sub-sections can clearly identify any issues that we have (so far)
failed to address herein.  

Here we describe the broad ``to do'' list that defines the roadmap for
all BH science, in addition to focusing on tasks needed for certain
science cases and also areas where the needs of this group may overlap
with roadmap plans from other groups.



\section{AGN Selection and Census}

Black Hole (BH) science will be a major theme for LSST covering
science topics ranging from the BH-fueled evolution of galaxies, to
lensed black holes as tools for cosmology, to small-scale physics of
the BHs themselves using variability and tidal disruptions as probes.
For all BH science with LSST the first step that must be taken is to
find the BHs themselves, whether through short-lived transient events
or longer-lived fueling in the context of active galactic nuclei
(AGN).  In that sense the primary goal for this roadmap for AGN
science is seemingly straightforward: identify efficient and complete
ways to pinpoint the location of black holes on the sky, and, ideally,
their distance/redshift from us.  In practice the devil is in the
details and the optimal solution to this problem may be different for
each of the LSST's BH science goals.

In the past, BH science could afford low efficiency in candidate
selection because of spectroscopy: even if the majority of targets
were not BHs, at least that was known.  Today, while we can hope for a
significant amount of spectroscopy from DESI and/or similar projects,
it is clear that we will not obtain spectroscopy for tens of millions
of candidate BHs.  Thus we need to concentrate on going it alone. In
simplest terms this means identifying BHs with both high completeness
and high efficiency.

Unfortunately the optimal methods for doing do are not expected to be
generic for all BH science.  Broadly speaking LSST BH science can be
divided into the need to identify four types of BHs: unobscured
quasars, obscured quasars, lower-luminosity AGNs, and transient BH
fueling events.  This is further complicated by the fact that objects
in each category may require somewhat different methods as a function
of redshift and/or by (close) spatial separation.

For example, selection of unobscured quasars can take full advantage
of the LSST data set: multi-band, multi-epoch optical photometry and
astrometry, whereas selection of obscured quasars necessarily relies
on supporting data from outside LSST.  Lower-luminosity AGNs will
further need to be weeded from (the far more abundant) inactive
galaxies, and identification of transient-fueled BHs will be limited
by the number of epochs of detection.  Each of these cases do have a
similar task need: developing new tools and applying them to simulated
data.

For unobscured quasars we must optimize the identification of BHs using
\begin{itemize}
\item colors
\item variability
\item astrometry
\end{itemize}
---ideally all three simultaneously.  Moreover, we will have to
consider the impact of star-galaxy separation (can we use the same
algorithm(s) regardless of morphology?) and the evolution of these
properties with redshift.

For obscured quasars, we will necessarily be reliant upon information
from other sources to confirm the ``active'' nature of what will
otherwise appear as normal galaxies (or not appear at all) in LSST.

Without a doubt one of the single most important to do items in terms
of LSST AGN selection is to run simulations that use all 6 LSST bands,
covering as much area and as many epoch as possible.  Such simulations
should also include as much physics and empirical correlation as
possible.  This includes: luminosity-dependence of emission features,
magnitude-color correlations, variability physics, astrometric errors,
differential chromatic refraction, nuclear vs. host galaxy luminosity
correlation [and its relationship to morphology], star-galaxy
separation, lensing probability, broad absorption lines and dust
reddening (intrinsics, host galaxy, and intervening). 

Test simulations against real data.

A goal for LSST should be for every science working group to give a
probability for {\em every} LSST object that it is an object under
their umbrella.  This information can then be fed back into the
classification schemes developed by each group.  E.g., an object that
the AGN group gives a 30\% chance of being an AGN might be downgraded
if the SNe group flags it as an SN at 99\%.

\subsubsection{Color Selection}

Color-selection by itself is certainly the most mature of the avenues
for identifying AGNs.  Any application of modern statistical
techniques such as described by [take REFS from Tina's paper] will be
a major first step in the process.  

Specific to do items in this regard are to establish an agreed-upon
test bed of color data and have a ``bake off'' to determine which
methods are the most effective.  This could include (or be completely
based upon) simulated data.  

Moreover, it is unlikely that any final method(s) adopted for LSST AGN
selection will be completely color based, thus it will be important to
extend such efforts to multi-parameter selection methods using the
information from the next sections.

Isolation of {\em nuclear} colors through difference imaging may be
crucial for low-luminosity systems (which will provide the bulk of new
LSST discoveries).

\subsubsection{Lack of Proper Motion/Astrometry}

While certain quasars and stars can have very similar colors, the fact
that quasars do not have proper motions as do Galactic stars has long
been used as a discriminant.  That will be no different for LSST.
However, even with its precise astrometry, LSST will need to
distinguish between objects that are apparently moving and those that
are truly moving and do so as a function of apparent magnitude.
Simulated data can be used to start this process.

Another task that can begin now is to extend algorithms that use USNO
data as a time baseline extension to new data from GAIA.  Stripe 82
data can be used as an LSST-like testbed for this.

A more recent approach has been to take advantage of differential
chromatics refraction of AGNs (Kaczmarczik et al. 2009).  In short
this procedure makes use of the astrometric offset of an emission line
object from that expected (in the astrometric solution) for a
power-law source.  Peters et al. (2015) have developed a formalism for
including this information with color information, but more work is
needed to fully leverage this resource.

\subsubsection{Selection by Variability}

In many ways variability will be the cornerstone of object
classification for LSST.  However, variability by itself is unlikely
to be a panacea.  Even for luminous quasars, it has been shown that
variablity combined with colors works better for selection than
variability alone (Peters et al. 2015).  

Moreover, for low-luminosity AGNs which are expected to have the most
variable nuclei, increasing contamination from the host galaxy will
compromise variability-selection methods if insufficient care is
taken.  Thus one clear to do item for variability selection is to team
up with other groups with more expertise in difference imaging in
order to isolate the variability properties of the {\em nuclear}
emission.

The next most crucial step for variability analysis may well be
determining how to make use of {\em all} of the data.  Specifically,
most current investigations looking at variability only use one
photometric bandpass, whereas LSST will have 6.  For the $gri$ data,
it may be sufficient to apply a median-color based offset and treat
the (non-simultaneous) data as one (e.g., by kriging the data
together) with better time resolution than would be available with
just a single band.  The fact that quasars are systematically bluer
when brighter will add a complication to such efforts, however.  For
the $z$ and $y$ data, low S/N may make it difficult to treat these
measurements equivalently to $gri$.  Moreover, the spatial (and thus
time) separation of $g$ and $y$ could actually degrade the accuracy of
a merged light curve.  For the $u$-band data, low S/N and sampling of
the Ly-$\alpha$ forest rather than the quasar continuum at
high-redshift will add complications.  These are issues that can be
further investigated with existing (and ongoing) data sets such as
Stripe 82, DES, and Pan-STARRS.

Lastly, unlike magnitudes which are uniformly measured for all
objects, light curves are difficult to analyze in a non-parametric
fashion.  A functional form (or forms) must be agreed upon and it must
be realized that without an accurate redshift, comparison of
variability parameters often compares very different rest frames.
$A$-$\gamma$ is certainly the simplest parameterization, but the
damped random walk (DRW) is currently the most popular.  Further work
is needed to determine if these are sufficiently accurate to use or if
a different parameterization might be more effective.

Work with SDSS Stripe 82 has allowed some early work in this
direction.  More still needed, possibly taking advantage also of
Kepler, SDSS-RM, and/or OzDES data.

A/Gamma vs. DRW

observed vs. rest

ugrizy combination


\subsubsection{Combination with Multiwavelength Data}

This can be considered more generally as ``other facilities'' as some
data (e.g., Euclid) may also be in the optical.


\subsubsection{Photometric Redshifts}

Photo-z.  Need to reconcile high-L and low-L methods.  Full PDFs.

\subsubsection{Expected Number of AGNs}

Need to update HRH07

\subsection{Luminosity Function}


Pre-write QLF and clustering papers based on sims to show expected
gains if Universe model in sims is correct.

Once found, need to know completeness, efficiency, volume searched.


Needed for other things too, but clearly here:
Precursor observations
E.g., spectroscopy in DDFs and/or Stripe 82 to serve as truth tables
for completeness and efficiency and allow for early RM science.


\subsection{Clustering}

Pre-write QLF and clustering papers based on sims to show expected
gains if Universe model in sims is correct.


\subsection{Multiwavelength Physics}

Obscured

Some invisible to optical alone but discoverable through combination
with multi-wavelength data.  So, need to develop formalism for doing
this.

\subsection{Variability}

Var -- might want different parameterization for physics than for
selection.  

time-domain science.  Parametrize variability




\subsection{Transient Fueling Events}

\subsection{Gravitational Lenses}








