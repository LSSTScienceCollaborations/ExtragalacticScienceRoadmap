% LSST Extragalactic Roadmap
% Chapter: Introduction
% First draft by 

\chapter[Introduction]{Introduction}
\label{ch:intro}

{\justify
The Large Synoptic Survey Telescope (LSST) is a wide-field, ground-based
observatory designed to image a substantial fraction of the sky in six optical
bands every few nights. 
The observatory will operate for at least a decade, allowing 
stacked images to detect galaxies to redshifts well beyond unity. LSST and
its Deep-Wide-Fast and Deep Drilling Field surveys will meet 
the requirements of a broad range of science goals in astronomy, astrophysics and cosmology
\citep{ivezic2008a}. 
The LSST ranked first among large ground-based initiatives in the
2010 National Academy of Sciences decadal survey in astronomy and astrophysics \citep{nrc2010a},
and will begin survey operations early in the next decade.
This document, the {\it LSST Galaxies Science Roadmap}, outlines critical preparatory research efforts needed 
to leverage fully the power of LSST for extragalactic science. 

In 2008, eleven separate quasi-independent science collaborations were formed to
focus on a broad range of topics in astronomy and cosmology that the LSST could
address. Members of these collaborations have proven instrumental in helping to
develop the science case for LSST (encapsulated in the LSST Science Book;
\citealt{LSSTSciBook}), 
refine the concepts for the survey and for the data processing, and educate
other scientists and the public about the promise of this unique observatory.

The Dark Energy Science Collaboration (DESC) has taken the
next logical step beyond the Science Book. They identified they most critical
challenges the community will need to overcome
to realize the potential of LSST for
measuring the nature and effects of dark energy. The
DESC looked at five complementary
techniques for tackling dark energy, and outlined high-priority tasks for their
Science Collaboration during construction. The DESC designated sixteen 
new and existing working
groups to coordinate the work. The DESC documented these efforts
in a 133-page white paper \citep{LSSTDESC}. The DESC white
paper provides a guide for investigators looking for ways to contribute to the
overall DESC preparatory science effort, 
indicates clearly the importance of the advance work, and 
connects individual research projects together into a broader 
enterprise to enable dark energy science with LSST.

Following the lead of DESC, several other LSST community organizations 
including the AGN, Milky Way and Local Volume, and Solar System Science
Collaborations,
started to develop Roadmaps to outline critical preparatory tasks in
their science domains.
This document, led by members of the LSST Galaxies Science Collaboration, 
acts as a Roadmap for extragalactic science covering
galaxy formation and evolution writ large, the influence of dark matter structure
formation on the properties of galaxy populations, and the impact of supermassive
black holes on their host galaxies.
This Roadmap identifies the major high-level
science themes of these investigations, outlines how complementary techniques
will contribute, and identifies areas where advance work will prove essential. For this
preparatory work, the {\it LSST Galaxies Science Roadmap} emphasizes areas that are not adequately 
covered in the DESC Roadmap. 

Chapter \ref{ch:science_background} gives a brief summary of the LSST galaxies science background.
Many of the themes and projects are already set out in the LSST Science Book, 
which provides more details for many of the science investigations. 
Chapter \ref{ch:task_lists} presents preparatory science tasks for 
extragalactic science with LSST, organized by science topic.
Cross-references between complementary tasks in different science topics are
noted throughout the document.
The science task list content assumes that the work plan of the DESC will be executed
and that the resulting software and other data products resulting from the DESC
efforts will be made available to the other science collaborations.
}
\let\cleardoublepage\clearpage

