
% LSST Extragalactic Roadmap
% Chapter: science_background 
% Section: galaxies
% First draft by 

\section{Science Background: Galaxies}
\label{sec:sci:gal:bkgnd}

Galaxies represent fundamental astronomical objects
outside our own Milky Way. 
The large luminosities of galaxies enable their 
detection to extreme distances, providing abundant
and far-reaching probes into the depths of the universe.
At each epoch in cosmological history, the color
and brightness distributions of the galaxy population
reveal how stellar populations form with time and
as a function of galaxy mass. The progressive mix of
disk and spheroidal morphological components of 
galaxies communicate the relative importance of
energy dissipation and collisionless processes
for their formation.
Correlations between internal galaxy properties and
cosmic environments indicate
the ways the universe nurtures galaxies as they form.
The evolution of the
detailed characteristics of galaxies over cosmic time
reflects how fundamental astrophysics
operates to generate the rich variety of 
astronomical structures observed today.

Study of the astrophysics of galaxy formation represents
a vital science of its own, but the ready
observability of galaxies critically enables a host of
astronomical experiments in other fields. 
Galaxies act as the semaphores of the
universe, encoding information about
the development of large scale
structures and the mass-energy budget of the
universe in their spatial distribution. The mass distribution
and clustering of galaxies reflect essential
properties of dark matter, including potential
constraints on the velocity and mass of particle candidates.
Galaxies famously host supermassive black holes, 
and observations of active galactic nuclei provide
a window into the high-energy astrophysics of black hole
accretion processes. The porous interface between the
astrophysics of black holes, galaxies, and 
dark matter structures allows for astronomers to 
achieve gains in each field using the same datasets.

The Large Synoptic Survey Telescope (LSST) will provide a
digital image of the southern sky in six bands ($ugrizy$).
The area ($\sim18,000~\mathrm{deg}^2$) and depth 
($r\sim24.5$ for a single visit, $r\sim27.5$ coadded) of
the survey will enable research of such breadth
that LSST may influence essentially all extragalactic 
science programs that rely primarily on photometric data.
For studies of galaxies, LSST provide both an unequaled 
catalogue of billions of extragalactic sources and high-quality 
multiband imaging of individual objects. This section of
the {\it Extragalactic Roadmap} presents scientific
background for studies of these galaxies with LSST to provide a
context for considering how the astronomical community can
best leverage the catalogue and imaging datasets and for
identifying any required preparatory science tasks.

LSST will begin science operations during the next decade,
more than twenty years after the start of the Sloan
Digital Sky Survey \citep{york2000a} and subsequent precursor surveys
including PanSTARRS \citep{kaiser2010a}, the Subaru
survey with Hyper Suprime-Cam \citep{miyazaki2012a}, and the Dark
Energy Survey \citep{flaugher2005a}. Relative to these prior
efforts, extragalactic science breakthroughs
generated by LSST will likely benefit from its increased area, source
counts, and statistical samples, the constraining power of the
six-band imaging, and the survey depth and image quality. The following
discussion of LSST efforts focusing on the astrophysics of galaxies
will highlight how these features of the survey enable new science
programs.



\subsection{Star Formation and Stellar Populations in Galaxies}
\label{sec:sci:gal:bkgnd:stars}

Light emitted by stellar populations will
provide all the direct measurements made by
LSST. This information will be filtered through
the six passbands utilized by the survey 
($ugrizy$), providing constraints on the
rest-frame ultraviolet SEDs of galaxies to
redshift $z\sim6$ and a probe of rest-frame
optical spectral breaks to $z\sim1.5$. By
using stellar population synthesis modeling,
these measures of galaxy SEDS will enable 
estimates of the redshifts, star formation rates,
stellar masses, dust content, and 
population ages for potentially 
billions of galaxies. In the context of previous
extragalactic surveys, LSST
will enable new advances in our understanding
of stellar populations in galaxies by contributing
previously unachievable statistical power and an
areal coverage that samples the rarest cosmic
environments.

A variety of ground- and space-based observations
have constrained the
star formation history of the universe over the
redshift range that LSST will likely probe
\citep[for a recent review, see][]{madau2014a}.
The statistical power of LSST will improve our
knowledge of the evolving UV luminosity function,
luminosity density, and cosmic
star formation rate. The LSST observations can
constrain how the astrophysics of gas
cooling within dark matter halos, the efficiency
of molecular cloud formation and the star formation
within them, and
regulatory mechanisms like supernova and radiative
heating give rise to these statistical features
of the galaxy population. While measurement of
the evolving UV luminosity function can
help quantify the role of these 
astrophysical processes, the ability of LSST
to probe vastly different cosmic environments
will also allow for the robust quantification of any
changes in the UV luminosity function with
environmental density, and an examination of 
connections between environment and the fueling
of star formation.

Optical observations teach us about
the established stellar content of galaxies.  
For stellar populations older than $\sim100$ million
years, optical observations provide 
sensitivity to the spectral breaks near a
wavelength of $\lambda\approx4000\AA$ in the 
rest-frame related to absorption in the
atmospheres of mature stars. 
Such observations help constrain
the amount of stellar mass in galaxies. For
passive galaxies that lack vigorous star formation,
these optical observations reveal
the well-defined ``red sequence'' of
galaxies in the color-magnitude plane
that traces the succession of
galaxies from recently-merged spheroids
to the most massive systems at the
centers of galaxy clusters. For blue,
star-forming
galaxies, optical light can help
quantify the relative contribution of
evolved stars to total galaxy luminosity, 
and indeed has
led to the identification of a well-defined
locus of galaxies in the parameter space of
star formation rate and stellar mass 
\citep[e.g.,][]{noeske2007a}. This
relation, often called the ``star-forming
main sequence'' of galaxies, indicates that
galaxies of the same stellar mass typically
sustain a similar star-formation rate. 
Determining the
physical or possibly statistical 
origin of the relation remains an active
line of inquiry, guided by recently improved
data from Hubble Space Telescope over the
$\sim0.2$ deg$^{-2}$ Cosmic Assembly Near-Infrared
Deep Extragalactic Survey 
\citep{grogin2011a,koekemoer2011a}. While 
LSST will be comparably limited in redshift 
selection, its $~30,000$ times larger area
will enable a much fuller sampling of the
star formation--stellar mass plane, allowing
for a characterization of the distribution
of galaxies that lie off the main sequence
that can help discriminate between phenomenological
explanations of the sequence.

\subsection{Galaxies as Cosmic Structures}
\label{sec:sci:gal:bkgnd:structures}

The structural properties of galaxies arise from
an intricate combination of important astrophysical
processes. The gaseous disks of galaxies require
substantial energy dissipation while depositing
angular momentum into a rotating structure. These
gaseous disks form stars with a
surface density that declines exponentially with
galactic radius, populating stellar orbits that
differentially rotate about the galactic center and
somehow organize into spiral features.
Many disk galaxies contain (pseduo-)bulges that form through
a combination of violent relaxation and orbital dynamics.
These disk galaxy features contrast with systems where
spheroidal stellar distributions dominate the galactic
structure. Massive ellipticals form through galaxy
mergers and accretions, and manage to forge a regular
sequence of surface density, size, and stellar velocity
dispersion from the chaos of strong gravitational
encounters. Since these astrophysical
processes may operate with great
variety as a function of galaxy mass and
cosmic environment, LSST will revolutionize studies
of evolving galaxy morphologies by providing enormous
samples with deep imaging of exquisite quality.

The huge sample of galaxies provided by LSST will
provide a definitive view of how the sizes and
structural parameters of disk and spheroidal systems
vary with color, stellar mass, and luminosity. 
Morphological studies will employ at least two
complementary techniques for quantifying the 
structural properties of galaxies. Bayesian
methods can yield multi-component
parameterized models for all the galaxies
in the LSST sample, including the quantified
contribution of bulge, disk, and
spheroid structures to the observed galaxy
surface brightness profiles. The parameterized
models will supplement non-parametric measures
of the light distribution including the
Gini and M20 metrics that quantify the surface
brightness uniformity and spatial moment of
dominant pixels in a galaxy image \citep{abraham2003a,lotz2004a}.
Collectively, these morphological measures provided
by analyzing the LSST imaging data will enable
a consummate determination of the relation between
structural properties and other features of
galaxies over a range of galaxy mass and luminosity
previously unattainable.

While the size of the LSST sample supplies the
statistical power for definitive morphological studies,
the sample size also enables the identification of rare
objects. This capability will benefit our efforts for
connecting the distribution of galaxy morphologies to their
evolutionary origin during the structure formation process,
including the formation of disk galaxies.
The emergence of ordered disk galaxies remains a hallmark
event in cosmic history, with so-called ``grand design''
spirals like the Milky Way forming dynamically cold, thin
disks in the last $\sim10$ Gyr. Before thin disks emerged,
rotating systems featured ``clumpy'' mass distributions with
density enhancements
that may originate from large scale gravitational instability.
Whether the ground-based LSST can effectively probe
the exact timing and duration of the transition from
clumpy to well-ordered disks remains 
unknown, but LSST can undoubtedly contribute studying the
variation in forming disk structures at the present day.
Unusual objects, such as the UV luminous local galaxies identified
by \citet{heckman2005a} that display physical features analogous to 
Lyman break galaxies at higher redshifts, may provide a 
means to study the formation of disks in the present day
under rare conditions only well-probed by the sheer size
of the LSST survey.

Similarly, the characterizing the extremes of the
massive spheroid population can critically inform
theoretical models for their formation. For instance,
the most massive galaxies at the centers of galaxy clusters
contain vast numbers of stars within enormous stellar
envelopes. The definitive LSST sample can capture enough
of the most massive, rare clusters to quantify the 
spatial extent of these galaxies at
low surface brightnesses, where the bound stellar
structures blend with the intracluster light of
their hosts. Another research area the LSST data
can help address regards the central densities of local
ellipticals that have seemingly decreased compared with
field ellipticals at higher redshifts. The transformation
of these dense, early ellipticals to the spheroids in the
present day may involve galaxy mergers and environmental
effects, two astrophysical processes that LSST can characterize
through unparalleled statistics and environmental probes.
By measuring the
surface brightness profiles of billions of 
ellipticals LSST can determine whether any such dense
early ellipticals survive to the present day, whatever
their rarity.

Beyond the statistical advances enabled by LSST and the
wide variation in environments probed by a survey
of half the sky, the image quality of LSST will permit
studies of galaxy structures in the very low surface
brightness regime. Observational
measures of the outer most regions of thin disks can constrain
how such disks ``end'', how dynamical effects might truncate 
disks, and whether some disks smoothly transition into stellar
halos. LSST will provide such measures and  help quantify the
relative importance the physical effects that influence the
low surface brightness regions in disks. Other galaxies
have low surface brightnesses throughout their stellar 
structures, and the image quality and sensitivity 
of LSST will enable the most complete census
of low surface brightness galaxies to date. LSST will provide
the best available constraints on the extremes of disk
surface brightness, which relates to the extremes of
star formation in low surface density environments.

The ability of LSST to probe low surface brightnesses
also allows for characterization of stellar halos that
surround nearby galaxies. Structures in stellar halos,
from streams to density inhomogeneities, originate
from the hierarchical formation process and their
morphology provides clues to the formation history
on a galaxy-by-galaxy basis \citep{bullock2005a,johnston2008a}.
Observations with small telescopes \citep{martinez-delgado2008a,abraham2014a}
have already 
demonstrated that stellar halo structures display interesting
variety
\citep[e.g.,][]{van_dokkum2014a}. 
LSST, with its unrivaled entendue, can help build a statistical
sample of stellar halos and cross-reference their morphologies
with the observed properties of their central galaxies. Such
studies may determine whether the formation histories reflected
in the structures of halos also influence galaxy colors or
morphological type. The
examination of stellar halos around external galaxies may
also result in the identification of small mass satellites
whose sizes, luminosities, and abundances can constrain
models of the galaxy formation process on the extreme
low-mass end of the mass function.

\subsection{Probing the Extremes of Galaxy Formation}
\label{sec:sci:gal:bkgnd:rare}

The deep, multiband imaging LSST provides over an enormous
area will enable the search for galaxies that form in the
rarest environments, under the most unusual conditions,
and at very early times. By probing the extremes of
galaxy formation, the LSST data will push our theoretical
understanding of the structure formation process.

The rarest, most massive early galaxies may form in 
conjunction with the supermassive black holes that
power distant quasars. LSST can use the same
types of color-color selections to identify extremely
luminosity galaxies out to redshift $z\sim6$, and
monitor whether the stellar mass build-up in these
galaxies tracks the accretion history of the most
massive supermassive black holes. If stellar mass
builds proportionally to black hole mass in quasars,
then very rare luminous star forming galaxies at
early times may immediately proceed the formation
of bright quasars. LSST has all the requisite
survey properties (area, mutliband imaging, and
depth) to investigate this long-standing problem.

The creation of LSST Deep Drilling fields will
enable a measurement of the very bright end
of the high-redshift galaxy luminosity function.
Independent determinations of the distribution of 
galaxy luminosities at $z\sim6$ show substantial
variations at the bright end. The origin of
the discrepancies between various groups remains
unclear, but the substantial cosmic variance expected
for the limited volumes probed and the intrinsic
rarity of the bright objects may conspire to
introduce large potential differences between
the abundance of massive galaxies in different
areas of the sky. Reducing this uncertainty requires
deep imaging over a wide area, and the LSST Deep Drilling
fields satisfy this need by achieving sensitivities
beyond the rest of the survey. 

Lastly, the spatial rarity of extreme objects discovered
in the wide LSST area may reflect an intrinsically
small volumetric density of objects or the short duration
of an event that gives rise to the observed properties of the
rare objects. Mergers represent a critical class 
of short-lived epochs in the formation histories of
individual galaxies. Current determinations of the evolving numbers
of close galaxy pairs or morphological indicators of
mergers provide varying estimates for the
redshift dependence of the galaxy merger rate 
\citep[e.g.,][]{conselice2003a,kartaltepe2007a,lotz2008a,lin2008a,robotham2014a}.
The identification of merging
galaxy pairs as a function of separation, merger
mass ratio, and environment in the LSST data will enable
a full accounting of how galaxy mergers influence
the observed properties of galaxies as a function of
cosmic time. 

\subsection{Science Book}
\label{sec:sci:gal:bkgnd:scibook}

The contents of the
Galaxies Chapter 9 of the Science Book (\citealt{LSSTSciBook}).

\begin{enumerate}
\item Measurements, Detection, Photometry, Morphology
\item Demographics of Galaxy Populations
\begin{itemize}
\item Passively evolving galaxies
\item High-redshift star forming galaxies
\item Dwarf galaxies
\item Mergers and interactions
\end{itemize}
\item Distribution Functions and Scaling Relations
\begin{itemize}
\item Luminosity and size evolution
\item Relations between observables
\item Quantifying the Biases and Uncertainties
\end{itemize}
\item Galaxies in their Dark-Matter Context
\begin{itemize}
\item Measuring Galaxy Environments with LSST
\item The Galaxy-Halo Connection
\item Clusters and Cluster Galaxy Evolution
\item Probing Galaxy Evolution with Clustering Measurements
\item Measuring Angular Correlations with LSST, Cross-correlations
\end{itemize}
\item Galaxies at Extremely Low Surface Brightness
\begin{itemize}
\item Spiral Galaxies with LSB Disks
\item Dwarf Galaxies
\item Tidal Tails and Streams
\item Intracluster Light
\end{itemize}
\item Wide Area, Multiband Searches for High-Redshift Galaxies
\item Deep Drilling Fields
\item Galaxy Mergers and Merger Rates
\item Special Populations of Galaxies
\item Public Involvement
\end{enumerate}



 

