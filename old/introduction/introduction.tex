% LSST Extragalactic Roadmap
% Chapter: Introduction
% First draft by 

\chapter[Introduction]{Introduction}
\label{ch:intro}

The Large Synoptic Survey Telescope (LSST) is a wide-field, ground-based
telescope, designed to image a substantial fraction of the sky in six optical
bands every few nights. It is planned to operate for a decade allowing the
stacked images to detect galaxies to redshifts well beyond unity. The LSST and
the survey are designed to meet the requirements (Ivezic \& the LSST Science
Collaboration 2011) of a broad range of science goals in astronomy, astrophysics
and cosmology.  The LSST was the top-ranked large ground-based initiative in the
2010 National Academy of Sciences decadal survey in astronomy and astrophysics,
and is on track to begin the survey early in the next decade.

In 2008, eleven separate quasi-independent science collaborations were formed to
focus on a broad range of topics in astronomy and cosmology that the LSST could
address. Members of these collaborations have been instrumental in helping to
develop the science case for LSST (encapsulated in the LSST Science Book), to
refine the concepts for the survey and for the data processing, and to educate
other scientists and the public about the promise of this unique observatory.

The Dark Energy Science Collaboration (DESC) has taken the
next logical step beyond the science book. They identified they most critical
challenges that will need to be overcome to realize LSST’s potential for
measuring the effects of Dark Energy. They looked at five complementary
techniques for tackling dark energy, and outlined high-priority tasks for the
science collaboration during construction. They designated sixteen working
groups (some of which already existed) to coordinate the work. This roadmap has
been documented in a 133-page white paper (arxiv.org/abs/1211.0310). The white
paper provides a guide for investigators looking for ways to contribute to the
overall investigation. It may help in efforts to obtain funding, because it
provides clear indication of the importance of the advance work and how the
pieces fit together.

The investigation of Dark Energy is only one topic for LSST. It is important to
develop similarly concrete roadmaps for work in other areas.  After some
discussion among the collaborations, it appears useful in some cases for
different science collaborations to join forces on a single whitepaper. This is
particularly true for topics that involve observations of distant galaxies. With
the advent of the DESC, some of the science goals of the large-scale-structure,
weak-lensing, and strong-lensing collaborations have found a new home. The
remaining science goals of those collaborations tend to be focused on galaxy
evolution and dark matter. Two other collaborations: AGN and Galaxies, also have
those topics as major themes. This roadmap identifies the major high-level
science themes of these investigations, outlines how complementary techniques
will contribute, and identifies areas where advance work is essential. For this
advance work, the emphasis is on areas that are not adequately covered in the
DESC roadmap. 
%As convenient shorthand, we use the acronym GALLA (Galaxies, AGN, Lensing
%Large-scale Structure and Astro-informatics) joint roadmap of the overlapping
%science collaborations.

Chapter \ref{ch:science_background} gives a brief summary of the science background.
Many of the themes and projects are already set out in the Science Book, where more
detail is provided for many of the science investigations. 
Chapter \ref{ch:task_lists} presents preparatory science tasks for 
Extragalactic science with LSST. These tasks are organized by science topic.
The science task list content assumes that the work plan of the DESC will be executed
and that the resulting software and other data products resulting from the DESC
efforts will be made available to the other science collaborations.
%Chapter \ref{ch:roadmap}
%sets out the highest priority preparatory work to enable these investigations. 
%These tasks
%are laid out on the assumption that the work plan of the DESC will be carried out
%and that software and data products resulting from that work will be available to 
%other science collaborations. The Appendix \ref{ch:task_lists} organizes the tasks
%by science topic and desribes them in more detail.

