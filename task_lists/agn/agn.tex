\section{Active Galactic Nuclei}\label{sec:tasks:agn:intro} {\justify


Active Galactic Nuclei (AGN) phenomena enable us to understand the growth of supermassive black holes (BHs), understand aspects of galaxy evolution, probe the high redshift universe, and study other physical activity including accretion physics, jets, and magnetic fields.  
While AGN represent a distinct topic within the LSST Science Collaborations, the LSST
dataset will reveal some aspects of AGN science via their role as an
evolutionary stage of galaxies in addition to their ability to probe accretion physics around BHs.
The tasks listed here present preparatory science efforts connected with AGN study as a special
phase in galaxy evolution.


\begin{tasklist}{AGN}
\subsection{AGN Selection from LSST Data}
\tasktitle{AGN Selection from LSST Data}
\begin{task}
\label{task:agn:selection}
\motivation{
LSST multiband photometry may select Active Galactic Nuclei using a variety of different methods.  At optical and near infrared wavelengths, the distinctive colors of AGN
at particular redshifts enables their photometric selection.
The LSST data will therefore augment methods that rely on X-ray or radio activity, or the
identification of emission lines in spectroscopic data.
LSST will also open up, in a more practical way, the identification of AGN based on their variability.   
These LSST photometric, multiwavelength, and variability-selected samples may probe
unique aspects of AGN phenomena.
A better understanding of the AGN role in galaxy evolution requires 
that we understand how and why these selection methods include or exclude particular sources
or phases of AGN-galaxy co-evolution.
}
~\\
\activities{
The use of LSST as a single way to identify AGN and characterize their diversity of AGN
requires the development of selection criteria that can leverage the color, morphology,
and variability information available from LSST imaging alone.
A number of AGN surveys with input from multiple wavelength observation and spectra already
exist, and precursor work must utilize these surveys to determine 
whether AGN that prove difficult to identify via optical color selection will reveal
themselves through the additional parameters of morphology, variability, and/or the 
near-infrared data that LSST will provide.
}
~\\
\deliverables{Deliverables over the next several years from the activities described above include the following:
\begin{enumerate}
\item Creation of a cross-matched catalog of known AGN selected and verified using different methods.
\item Development of morphological parameters beyond star/galaxy separation and an understanding of the efficacy of LSST Level 2 data products for morphological selection of AGN.
\item Development of color selection criteria that accounts for morphology.
\item Understanding of AGN variability sensitivity given the nominal LSST cadence. 
\item Development of algorithms for color selection account for AGN variability.
\end{enumerate}
}
\end{task}
%\end{tasklist}


%\begin{tasklist}{T}
\subsection{AGN Host Galaxy Properties from LSST Data}
\tasktitle{AGN Host Galaxy Properties from LSST Data}
\begin{task}
\label{task:agn:host_galaxies}
\motivation{
Morphological characterizations from parameterized models, such as multiple-component 
\cite{sersic1968a} profiles, or non-parametric measures like CAS and Gini-M20 \cite{abraham1994a,conselice2000a,lotz2004a} can help identify mergining galaxies in the LSST data.
The ability of these techniques to characterize efficiently and accurately the
morphology of AGN host galaxies identified via their variability remains unproven.
}
~\\
\activities{
Simulated or model AGN host galaxies can characterize whether the
LSST Level 2 data will enable the measurement of morphological features associated with
AGN, as a function of host galaxy properties, AGN luminosity, and variability.
For each model galaxy, varying the central AGN luminosity will reveal the impact of
central source brightness on the recovery of morphological properties.
Existing data sets, such as Pan-STARRS, may help inform LSST about the range of
variability frequency and amplitude, and how these AGN properties may affect the
recovery of morphological properties in AGN host galaxies. 
}
~\\
\deliverables{Deliverables over the next several years from the activities described above include the following:
\begin{enumerate}
\item Characterization of the accuracy and precision afforded by the LSST dataset for the
recovery of basic morphology properties as a function of AGN brightness and wavelength.
\item Understaing of the effects of AGN brightness and variability on host-galaxy classification diagrams.
\end{enumerate}
}
\end{task}
%\end{tasklist}

\subsection{AGN Feedback in Clusters}
\tasktitle{AGN Feedback in Clusters}
\begin{task}
\label{task:agn:feedback_in_clusters}
\motivation{
Brightest Cluster/Group Galaxies (hereafter BCGs) represent the most massive galaxies in the local 
universe, residing at/near the centers of galaxy clusters/groups. 
BCGs contain the largest known supermassive BHs that can influence the host galaxy properties,
cluster gas, and other cluster members via the mechanical energy produced by their $>100$kpc 
scale jets (``AGN feedback'').
\\
The relative proximity of low-redshift galaxy clusters enable detailed studies of 
stars, gas, and AGN jets that may reveal the ramifications of AGN feedback. 
LSST will provide a large sample of moderate- to high-redshift clusters 
in which we can measure AGN feedback statistically. By combining X-ray, radio, and optical observations we can assess the average influence of BCG AGN on the hot intracluster medium (ICM) for different sub-populations \citep[e.g.,][]{stott2012a}.
}
~\\
\activities{
By assembling a multi-wavelength dataset (optical, X-ray, and radio) we can obtain the BCG mass, cluster mass, ICM temperature, and the mechanical power injected into the ICM. 
We can use this to study the interplay between the BCG, its black hole, and the cluster gas, 
providing an assessment of the balance of energies involved and a direct comparison with theoretical models of AGN feedback. 
SDSS has enabled this multi-wavelength analysis for a few hundred clusters at $z<0.3$,
but LSST cluster datasets to reach $z>1$.
Such studies hold implications for cosmological studies by helping to distinguish between
X-ray gas properties strongly influenced by AGN in addition to the
cluster mass.
}
~\\
\deliverables{Deliverables over the next several years from the activities described above include the following:
\begin{enumerate}
\item Investigation of the number of BCGs and the mass range of their clusters with redshift that LSST will likely observe.
\item Assessment of radio and X-ray data available for AGN Feedback studies (XCS, eROSITA, SKA-pathfinders, SUMSS, etc.).
\item Assessment of theoretical predictions expected for the multi-wavelength properties of
AGN host galaxies in clusters or groups (e.g., cosmological simulations such as EAGLE or more detailed single cluster studies).
\end{enumerate}
}
\end{task}
%\end{tasklist}

%\begin{tasklist}{T}
\subsection{AGN Variability Selection in LSST Data}
\tasktitle{AGN Variability Selection in LSST Data}
\begin{task}
\label{task:agn:variability}
\motivation{
Most AGN exhibit broad-band aperiodic, stochastic variability across the entire electromagnetic
spectrum on timescales ranging from minutes to years. Continuum variability arises in the accretion disk of the AGN, providing a powerful probe of accretion physics. 
The main LSST Wide-Fast-Deep (WFD) survey will obtain $\sim10^8$ AGN light curves (i.e., flux as a function of time) with $\sim1000$ observations ($\sim200$ per filter band) over 10 years. 
The Deep Drilling Fields will provide AGN lightcurves with much denser sampling for a small subset of the objects in the WFD survey. The science content of the lightcurves will critically depend on the exact sampling strategy used to obtain the light curves. For example, the observational uncertainty in determining the color variability of AGN will crucially depend on the interval between observations in individual filter bands. These concerns motivate a determinination of guidelines for an optimal survey strategy (from an AGN variability perspective) and a discovery
of possible biases and uncertainties introduced into AGN variability science as a result of the chosen survey strategy.}
~\\
\activities{
Study existing AGN variability datasets (SDSS Stripe 82, OGLE, PanSTARRS, CRTS, PTF + iPTF, Kepler, \& K2) to constrain a comprehensive set of AGN variability models. Generate and study simulations using parameters selected from these models using observational constraints, and determine the appropriateness of simulations for carrying out various types of AGN variability science including power spectrum models, quasi-periodic oscillation searches, and binary AGN models.}
~\\
\deliverables{Deliverables over the next several years from the activities described above include the following:
\begin{enumerate}
\item Observational constraints on AGN variability models.
\item MAF metrics quantifying the goodness of different survey strategies for AGN variability science.
\end{enumerate}
}
\end{task}
%\end{tasklist}

%\begin{tasklist}{T}
\subsection{AGN Photometric Redshifts from LSST Data}
\tasktitle{AGN Photometric Redshifts from LSST Data}
\begin{task}
\label{task:agn:photoz}
\motivation{
Given the large number of AGN that LSST will discover, 
many AGN will not receive follow-up with spectroscopic observations.
Photometric redshifts can provide relatively accurate redshifts for large numbers of galaxies,
but accurate photometric redshifts for AGN host galaxies remain challenging.
}
~\\
\activities{
Initial efforts include a comprehensive review of the state of the art in AGN host galaxy photo-$z$
determinations and an analysis of AGN vs. non-AGN galaxy photo-$z$ performance.
A comparison of model and/or observed AGN host SEDs with a matched set of
non-host galaxies at a variety of redshifts will help engineer color selection criteria for identifying AGN hosts, and whether variability can break photo-$z$ degeneracies. 
}
~\\
\deliverables{Deliverables over the next several years from the activities described above include the following:
\begin{enumerate}
\item Development of AGN host color selection criteria, and an identification of objects for which color selection might prove ambiguous or degenerate.
\item Exploration of multiwavelength, morphological, or variability information that might break photo-$z$ degeneracies.
\end{enumerate}
}
\end{task}
%\end{tasklist}

%\begin{tasklist}{T}
\subsection{AGN Merger Signature from LSST Data}
\tasktitle{AGN Merger Signature from LSST Data}
\begin{task}
\label{task:agn:mergers}
\motivation{
Understanding the role AGN play in galaxy evolution requires identifying AGN phenomena at all stages and in all types of galaxies. 
AGN host galaxies often show disturbed morphology, suggesting that the galaxy merger process may  trigger AGN activity. 
While the ``trainwrecks'' may prove easy to identify in the high-quality LSST data, the
identification of galaxies in other merger stages, such as ``pre-merger'' harassment, may be particularly hard to recognize.
Preliminary work needs to be done to understand how to identify mergers from the LSST data products and whether galaxy deblending and segmentation methods and procedures are adequate.
}
~\\
\activities{
Create simulated or identify real images that contain known galaxy mergers, including
systems with and without visible AGN.
Run the LSST software stack on these images, and
engineer metrics that describe/quantify 
the accurate detection of galaxy mergers with and without AGN.
}
~\\
\deliverables{Deliverables over the next several years from the activities described above include the following:
\begin{enumerate}
\item Improve the ability of the LSST Level 2 data to enable the detection of galaxy mergers with AGN.
\item Identify merging galaxy properties or categories that will prove challenging to measure using the LSST datasets.
\end{enumerate}
}
\end{task}
\end{tasklist}
}