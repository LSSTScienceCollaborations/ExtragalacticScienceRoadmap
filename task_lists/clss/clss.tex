\section{Clusters and Large Scale Structure}\label{sec:tasks:clss}  
{\justify
The cosmological process of galaxy formation inextricably links
together environment and large scale structure with the detailed
properties of galaxy populations. The extent of this connection
ranges from the scales of superclusters down to small groups. The
following preparatory science tasks focus on this critical connection
between galaxy formation, clusters, and large scale structure (LSS).

\begin{tasklist}{CLSS}
\subsection{Cluster and Large Scale Structure Sample Emulator}
\tasktitle{Cluster and Large Scale Structure Sample Emulator}
\begin{task}
\label{task:clss:emulator}
\motivation{
To prepare for galaxy group/cluster and LSS science with LSST,
the samples of cluster/group galaxies detected in a given range of
redshift, brightness, and color need to be estimated.
Group and
cluster populations must be identified in given ranges of redshift, richness, mass,
and other physical parameters.
}
~\\
\activities{
LSST has advanced simulations of its 10-year Wide Fast Deep survey
available from the Operations Simulator. The output databases can be
analyzed to determine the expected time-dependent depth of LSST
detection images at each sky location. These depths can be converted
into predicted numbers of galaxies as a function of redshift and
brightness \citep[e.g.,][]{awan2016a}.\\
~\\
To predict galaxy sample sizes as a function of physical parameters,
the ``raw'' predicted galaxy numbers can be
interfaced with semi-analytical models painted on large N-body
simulations. This approach can extend the
predictions for cluster/LSS samples
to include the observed properties of color, size, morphology,
and the physical properties of halo mass, stellar mass, and star formation
rate. The combination of these models will produce an LSST Cluster/LSS
Sample Emulator for understanding the detailed science return of
LSST for cluster and group science.}
~\\
\deliverables{%Deliverables over the next several years from the activities described above include the following:
~
\begin{enumerate}
\item Creation of a public LSST Extragalactic Sample Emulator with a
simple user interface, allowing for the estimation of sample sizes
detected by LSST as a function of redshift and 
physical parameters (e.g.~galaxy magnitudes, colors, size, morphology,
cluster richness, mass, temperature, etc.).
\end{enumerate}
}
\end{task}


\subsection{Identifying and Characterizing Clusters}
\tasktitle{Identifying and Characterizing Clusters}
\begin{task}
\label{task:clss:clusters}
\motivation{ LSST photometry will make it possible to search for and 
study the galaxy populations of distant clusters and proto-clusters over
cosmological volumes. These clusters are testbeds for theories of
hierarchical structure formation, intergalactic medium heating, metal
enrichment, and galaxy evolution. However, standard approaches for
identifying clusters, such as the red sequence method, will be hampered
by the limited wavelength coverage of LSST. For example, at $z \gtrsim
1.5$, near-IR photometry is required to identify systems with
Balmer/$4000$\AA\ breaks.
To maximize cluster science with LSST, new
techniques for cluster identification and the incorporate complementary
data from projects such as \emph{Euclid} or \emph{eROSITA} must be devised.
Exploration of optimal filter methods 
\citep[e.g.,][]{bellagamba2017a}
could improve cluster identification in LSST data.
}
~\\
\activities{ Using existing imaging datasets and simulations, algorithms
need to be developed and optimized to identify clusters at intermediate-
and high- redshift within the LSST footprint. Specifically, this work
should characterize the selection function, completeness, and
contamination rate for different cluster identification algorithms. Such
characterizations require realistic light-cone simulations spanning
extremely large volumes, so as to capture significant numbers
($\gg10,000$) of simulated galaxy clusters at high-redshift. Potential
algorithms to be tested include adaptations of RedMaPPer
\citep{rykoff2014a}, methods that search for galaxy
overdensities over a range of scales \citep[e.g.,][]{chiang2014a,wang2016a},
and 
methods designed to select clusters from joint optical/NIR/X-ray 
datasets. In parallel, a comprehensive census of the available 
multiwavelength data (specifically IR and X-ray imaging) is needed to
enable the new algorithms to be tested on real observational data, 
including multiwavelength datasets co-located with LSST Commissioning 
observations.  It will also be necessary to develop algorithms to 
homogenize the external datasets with LSST data at the pixel level, and
enable them to be served to the community.}
~\\
\deliverables{%Deliverables over the next several years from the activities described above include the following:
~
\begin{enumerate}
\item New and improved cluster identification algorithms that can be 
applied to LSST commissioning and survey data, and combined 
LSST/NIR/X-ray data.
\item Compilation of auxiliary data that will aide testing of algorithms 
to identify and characterize clusters, such as near-infrared (e.g. 
VISTA, UKIDSS), X-ray (e.g.~XMM-Newton, eROSITA, etc.), SZ (Planck, SPT, 
ACT), and radio (SKA and its pathfinders, SUMSS) within the LSST 
footprint.
\item Algorithms to match optimally and homogenize multiwavelength 
data with LSST commissioning and survey data.
\end{enumerate}
}
\end{task}

\subsection{Developing and Optimizing Measurements of Galaxy Environment}
\tasktitle{Developing and Optimizing Measurements of Galaxy Environment}
\begin{task}
\label{task:clss:environment}
\motivation{
Over the past decade, many studies have
shown that environment plays a important role in shaping galaxy
properties. For example, satellite galaxies in the local Universe
exhibit lower star formation rates, more bulge-dominated morphologies,
as well as older and more metal-rich stellar populations when compared
to isolated (or ``field'') systems of equivalent stellar mass 
\citep{baldry2006a,cooper2010a,pasquali2010a}.
Unlike spectroscopic surveys, LSST will lack the precise line-of-sight
velocity measurements to robustly identify satellite galaxies in
lower-mass groups, where the expected photo-$z$ precision will be much
coarser than the corresponding
the velocity dispersion of typical host halos.
Instead, LSST will likely be better suited to measure environment by
tracing the local galaxy density and identifying filaments. However,
LSST is unlike any previous photometric survey and may require new
approaches to measuring environment.
The challenge remains to find measures of local galaxy density
with the greatest sensitivity to the true underlying density field (or
to host halo mass, etc.), so as to enable analyses of environment's
role in galaxy evolution with LSST.
}
~\\
\activities{
Using mock galaxy catalogs created via
semi-analytic techniques, tracers of local
galaxy density (i.e.~``environment'') measured on mock LSST
photometric samples will be compared to the underlying real-space density of galaxies
or host halo mass. In addition to testing existing density
measures, such as $N^{\rm th}$-nearest-neighbor distance and counts in
a fixed aperture, new measures will be explored that may be better
suited to LSST. For each measure, the impact of
increasing survey depth and photo-$z$ precision over the course of the
survey will be examined.}
~\\
\deliverables{%Deliverables over the next several years from the activities described above include the following:
~
\begin{enumerate}
\item With an improved understanding of the
strengths and weaknesses of different environment measures as applied
to LSST, this effort will yield code to measure local galaxy density
(likely in multiple ways) within the LSST dataset.
\item Creation of methods to generate galaxy environmental measures as Level 3 data products for use by the entire project. 
\end{enumerate}
}
\end{task}

\subsection{Enabling and Optimizing Measurements of Galaxy Clustering}
\tasktitle{Enabling and Optimizing Measurements of Galaxy Clustering}
\begin{task}
\label{task:clss:clustering}
\motivation{
Contemporary galaxy surveys have
transformed the study of large-scale structure, enabling high
precision measurements of clustering statistics. The correlation
function provides a fundamental way to characterize the galaxy
distribution. The dependence of clustering on galaxy properties and
the evolution of clustering provide fundamental constraints on
theories of galaxy formation and evolution. Interpreting these
measurements provides crucial insight into the relation between
galaxies and dark matter halos. Understanding how galaxies relate to
the underlying dark matter is essential for optimally utilizing
the large-scale distribution of galaxies as a cosmological probe.
}
~\\
\activities{
Support work to define and characterize
the upcoming galaxy samples from LSST to enable clustering
measurements. Several distinct sets of information need to
be made available or be calculable from pipeline data. Such
requirements include a detailed understanding of any selection effects
impacting the observed galaxies, the angular and radial completeness
of the samples, and the detailed geometry of the survey (typically
provided in terms of random catalogs that cover the full survey area).
One aspect
to address is how best to handle the large data sets involved
(e.g.~the LSST ``gold'' galaxy sample will include about $4$~billion
galaxies over $18$,$000$~square degrees). Another issue is the development
of a
methodology to incorporate optimally the LSST photo-$z$ estimates with
the angular data to obtain ``2.5-dimensions'' for pristine clustering
measurements.\\
~\\
Further efforts will concern the development, testing, and optimization of
algorithms for measuring galaxy clustering using LSST data. 
These algorithms will be tested on realistic LSST mock catalogs, which
will also later serve as a tool for obtaining error estimates on the
measurements.
This endeavor overlaps with DESC-LSS working group efforts, and
requires cooperation of the DESC-PhotoZ working group and the Galaxies
Theory and Mock Catalogs working group.}
~\\
\deliverables{%Deliverables over the next several years from the activities described above include the following:
~
\begin{enumerate}
\item Guidance for LSST galaxy pipelines to include all the necessary
information for measurements of the correlation function and related
statistics to take place once LSST data is available. This deliverable
requires the engineering of ancillary products such as masks and maps of
completeness as a function of galaxy properties.
\item Development and refinement of techniques for measuring galaxy clustering of
large LSST galaxy samples. Together, these techniques will enable the
full potential of LSST data for combined studies of large scale structure and
galaxy formation to be realized.
\end{enumerate}
}
\end{task}


\subsection{Enabling Cluster Science through Robust Photometry and Photometric Redshifts}
\tasktitle{Enabling Cluster Science through Robust Photometry and Photometric Redshifts} 
\begin{task}
\label{task:clss:los}
\motivation{
Lines-of-sight through the cores of galaxy clusters and 
groups (hereafter "clusters") rank among the most challenging 
observations in which to identify robustly distinct extragalactic 
objects and to infer a reliable photometric redshift distribution
 $p(z)$ from the available data. 
Challenges that must be overcome include crowding of galaxies, a highly 
non-uniform background principally caused by the diffuse intracluster 
light, and the presence of numerous background galaxies including some 
that are highly distorted gravitational arcs.  Given that the LSST 
data will be accumulated over numerous epochs, there is significant 
potential for the detection of transients along these lines-of-sight.  
All of these complicating issues will be relevant to LSST commissioning data, 
Wide Fast Deep survey data, and the Deep Drilling Fields, and all will 
affect the basic processes of source detection and photometry.  Beyond 
photometry, further fundamental issues that require solutions include 
ensuring that photometric redshift algorithms are 
provided with spectral templates appropriate for
faint cluster members and prior information about the presence
of a galaxy cluster along the line-of-sight. 
Science goals affected by these issues include weak-lensing measurements of the 
mass-concentration relation as a function of halo mass and redshift, 
measuring the evolution of the galaxy luminosity function in clusters 
and in particular the faint end slope, identifying star-forming galaxies 
in clusters and their infall regions to probe the physics of quenching 
of star formation, measuring the evolution of brightest cluster 
galaxies, automated detection of strong-lensing clusters, and even 
potential identification of strongly-lensed transients including 
possible electromagnetic counterparts to gravitational wave sources.
}
~\\
\activities{
LSST will deliver the most information rich dataset ever in relation to 
the masses and internal structures of clusters and their infall regions.  
Moreover, the dataset can be enhanced significantly via the addition of 
data at other wavelengths, including X-ray, millimeter, and 
near-infrared.  To take full advantage of this opportunity, the 
fundamental issues of source detection, photometry, and photometric 
redshift inference along these crowded, challenging lines-of-sight must 
be solved. Relevant activities include testing the Level 2 
software pipeline on both simulated LSST observations of clusters and 
existing data including deep
observations of known clusters (e.g. from LoCuSS, Weighing the Giants, 
CCCP, CFHT-LS, HSC, etc.).  These tests will examine the ability of 
the Level 2 pipeline to deblend correctly these crowded lines-of-sight,
and determine whether the Level 2 algorithm will require modifications
or a Level 3 algorithm for clusters will be needed.
Further activities include collating, reviewing, and selecting 
appropriate cluster-specific galaxy spectral templates for deployment in 
photometric redshift codes.  
Another critical activity is to develop photometric 
redshift algorithms that account for the brightness and 
extent of X-ray emission, the overdensity of galaxies as a function of 
magnitude and color, any available spectroscopic redshifts, and the 
amplitude and extent of any SZ decrement/increment. 
Such algorithms will 
likely adopt a Bayesian hierarchical modeling approach to forward model 
the problem, and can be tested on simulated data based on numerical 
simulations and existing datasets (e.g. from XXL, XCS, HSC, LoCuSS, DES, 
and others). 
~\\~\\
This work links with efforts on deblending and
intracluster light, forward 
modeling of cluster and groups, environmental measures, cluster 
detection, auxiliary data, and work in the DESC Clusters Working Group 
via the determination of $p(z)$ for background galaxies.}
~\\
\deliverables{%Deliverables over the next several years from the activities described above include the following:
~
\begin{enumerate}
\item An algorithm that produces accurately deblended photometry of 
cluster cores at least to redshift $z = 1$ with LSST data and to higher 
redshifts in combination with near-IR data.
\item Cluster-specific galaxy spectral templates for use by photometric 
redshift codes, emphasizing samples drawn from spectroscopic surveys of 
high-redshift clusters.
\item A new cluster-specific photometric redshift algorithm that can be 
applied to a list of cluster detections based on LSST data, external 
data, or combined LSST/external datasets.
\end{enumerate}
}
\end{task}



\subsection{ Enabling Cluster Physics Through Forward Modeling of LSST 
Clusters and Groups}
\tasktitle{Enabling Cluster Physics Through Forward Modeling of LSST 
Clusters and Groups}
\begin{task}
\label{task:clss:cluster_fm}
\motivation{
Most of the interesting cluster and group physics from LSST and its
union with complementary surveys will be derived from studies that
explore the full range of halo mass relevant to groups and clusters
$M_{200}\simeq10^{13}-10^{15}M_\odot$ and to redshifts $z\gtrsim1$.
This mass and redshift range extends
beyond that used by cluster cosmologists (e.g., colleagues in DESC), 
who typically restrict attention to objects with masses 
$M_{200}>10^{14}M_\odot$ at $z<1$.
Extending the range probed will be very challenging from the point of 
view of systematic biases.  However, the requirement on controlling 
systematic biases for cluster/group physics is an order of magnitude 
less stringent than the nominal 1\% requirement for dark energy science.  
Arguably, $\sim10\%$ control of systematic biases in weak-lensing 
measurements of low redshift clusters ($\gtrsim2\times10^{14}M_\odot$) 
has already been achieved 
\citep{okabe2013a,applegate2014a,hoekstra2015a,okabe2016a}, giving 
considerable grounds for optimism.
}
~\\
\activities{
Broadly, cluster-based constraints on dark energy will be 
based on forward modeling the cluster population from a cosmological 
model and a number of scaling relations between halo mass and observable 
properties. The design of these algorithms and their selected
scaling relations will be predicated on deriving 
the most reliable dark energy constraints, and not necessarily for
learning the 
most cluster physics possible.  For example, the
scaling relations will 
be selected to have low scatter, and the internal 
structure and halo concentration of clusters may be treated as nuisance 
parameters despite their physical interest.
~\\~\\
The main activity in this task will therefore be to develop a code to 
forward model the combined LSST/multiwavelength dataset on a cluster 
population at a level of detail that preserves the structure of 
clusters and thus maximizes the available information about internal 
cluster physics.  
The model should connect cluster properties back to the halo mass function via the 
weak-lensing constraints that will be available from the LSST data, assuming
a fixed cosmological model.
Requirements missing from cosmological modeling codes include (1) simultaneous fitting 
of cluster scaling relations, density profile models to weak-shear 
profiles, and the mass-concentration relation of groups and clusters, 
and (2) simultaneous fitting of the weak-shear signal from clusters with 
overtly astrophysical parameters of interest such as the star-formation 
rate of clusters, and indicators of the merger history of clusters like 
X-ray morphology.  A public, multipurpose code would enable the
exploration of a broad range cluster science 
interests with LSST and supporting data. Other important activities 
will include testing the code on simulated light cone data, existing 
datasets (e.g. XCS, SPT, XXL, LoCuSS, HSC, DES, and others), and LSST 
commissioning data.
}
~\\
\deliverables{%Deliverables over the next several years from the activities described above include the following:
~
\begin{enumerate}
\item A robust Bayesian forward modeling code to constrain 
the physics of galaxies and hot gas in groups and clusters, tied 
directly to the halo mass function via weak-lensing from LSST.
\item Detailed tests on simulated and existing datasets.
\item Early science from LSST commissioning data, depending on the choice of 
commissioning fields.
\end{enumerate}
}
\end{task}
\end{tasklist}
}
