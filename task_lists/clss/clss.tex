\section{Clusters and Large Scale Structure}\label{sec:tasks:clss}  
{\justify
The cosmological process of galaxy formation inextricably links
together environment and large scale structure with the detailed
properties of galaxy populations. The extent of this connection
ranges from the scales of superclusters down to small groups. The
following preparatory science tasks focus on this critical connection
between galaxy formation, clusters, and large scale structure (LSS).

\begin{tasklist}{CLSS}
\subsection{Cluster and Large Scale Structure Sample Emulator}
\tasktitle{Cluster and Large Scale Structure Sample Emulator}
\begin{task}
\label{task:clss:emulator}
\motivation{
To prepare for galaxy group/cluster and LSS science with LSST,
the samples of cluster/group galaxies detected in a given range of
redshift, brightness, color, need to be estimated, along with the
group and
cluster populations identified in given ranges of redshift, richness, mass,
and other physical parameters.
}
~\\
\activities{
LSST has advanced simulations of its 10-year Wide Fast Deep survey
available from the Operations Simulator. The output databases can be
analyzed to determine the expected time-dependent depth of LSST
detection images at each sky location. These depths can be converted
into predicted numbers of galaxies as a function of redshift and
brightness \citep[e.g.,][]{awan2016a}.
\\
To predict galaxy sample sizes as a function of physical parameters,
the ``raw'' predicted galaxy numbers can be
interfaced with semi-analytical models painted on large N-body
simulations. This approach can extend the
predictions for cluster/LSS samples
to include the observed properties of color, size, morphology,
and the physical properties of halo mass, stellar mass, and star formation
rate. The combination of these models will produce an LSST Cluster/LSS
Sample Emulator for understanding the detailed science return of
LSST for cluster and group science.}
~\\
\deliverables{%Deliverables over the next several years from the activities described above include the following:
~
\begin{enumerate}
\item Creation of a public LSST Extragalactic Sample Emulator with a
simple user interface, allowing for the estimation of sample sizes
detected by LSST as a function of redshift and 
physical parameters (e.g.~galaxy magnitudes, colors, size, morphology,
cluster richness, mass, temperature, etc.).
\end{enumerate}
}
\end{task}


\subsection{Identifying and Characterizing Clusters}
\tasktitle{Identifying and Characterizing Clusters}
\begin{task}
\label{task:clss:clusters}
\motivation{
LSST photometry will make it possible to
search for and study the galaxy populations of distant clusters and
proto-clusters over cosmological volumes. These
clusters are testbeds for theories of
hierarchical structure formation,
intergalactic medium heating, metal enrichment, and galaxy evolution.
However, standard approaches for identifying clusters, such as the red
sequence method, will be hampered by the limited wavelength coverage
of LSST. 
For example, at $z \gtrsim 1.5$, near-IR photometry is required to 
identify systems with Balmer/$4000$\AA\ breaks. 
To maximize cluster science with LSST, new techniques
for cluster identification and the incorporate complementary data
from projects such as \emph{Euclid} or \emph{eROSITA} must be devised.
}
~\\
\activities{
Using existing imaging datasets and simulations, algorithms need to be
developed and optimised to identify clusters at intermediate
and high redshift within the LSST footprint. 
Specifically, this work should characterize the selection
function, completeness, and contamination rate for different cluster
identification algorithms. 
Such characerizations require realistic light-cone simulations spanning extremely
large volumes, so as to capture significant numbers ($\gg10,000$) of
simulated galaxy clusters at high redshift.
Potential algorithms to be tested include adaptations of RedMaPPer
\citep{rykoff2014a} as well as methods that search for galaxy
overdensities over a range of scales \citep[e.g.,][]{chiang2014a,wang2016a}.
In parallel, a comprehensive search for multiwavelength data
(specifically IR and X-ray imaging) is needed to aid in the search for
high-$z$ clusters and in the confirmation and characterization of
systems at all redshifts.}
~\\
\deliverables{%Deliverables over the next several years from the activities described above include the following:
~
\begin{enumerate}
\item Improved cluster
identification algorithms that can be applied to LSST data once
science operations commence.
\item Compilation of ancillary data
that will be helpful in cluster identification and characterization,
such as X-ray (e.g.~XCS, eROSITA, etc.), SZ (Planck, SPT, ACT) and
radio (SKA and its pathfinders, SUMSS), within the LSST footprint.
\end{enumerate}
}
\end{task}

\subsection{Developing and Optimizing Measurements of Galaxy Environment}
\tasktitle{Developing and Optimizing Measurements of Galaxy Environment}
\begin{task}
\label{task:clss:environment}
\motivation{
Over the past decade, many studies have
shown that ``environment'' plays a important role in shaping galaxy
properties. For example, satellite galaxies in the local Universe
exhibit lower star formation rates, more bulge-dominated morphologies,
as well as older and more metal-rich stellar populations when compared
to isolated (or ``field'') systems of equivalent stellar mass 
\citep{baldry2006a,cooper2010a,pasquali2010a}.
Unlike spectroscopic surveys, LSST will lack the precise line-of-sight
velocity measurements to robustly identify satellite galaxies in
lower-mass groups, where the expected photo-$z$ precision will be much
coarser than the corresponding
the velocity dispersion of typical host halos.
Instead, LSST will likely be better suited to measuring environment by
tracing the local galaxy density and identifying filaments. However,
LSST is unlike any previous photometric survey and may require new
approaches to measuring environment.
The challenge remains to find measures of local galaxy density
with the greatest sensitivity to the true underlying density field (or
to host halo mass, etc.), so as to enable analyses of environment's
role in galaxy evolution with LSST.
}
~\\
\activities{
Using mock galaxy catalogs created via
semi-analytic techniques, tracers of local
galaxy density (i.e.~''environment'') measured on mock LSST
photometric samples will be compared to the underlying real-space density of galaxies
or host halo mass. In addition to testing existing density
measures, such as $N^{\rm th}$-nearest-neighbor distance and counts in
a fixed aperture, new measures will be explored that may be better
suited to LSST. For each measure, the impact of
increasing survey depth and photo-$z$ precision over the course of the
survey will be examined.}
~\\
\deliverables{%Deliverables over the next several years from the activities described above include the following:
~
\begin{enumerate}
\item With an improved understanding of the
strengths and weaknesses of different environment measures as applied
to LSST, this effort will yield code to measure local galaxy density
(likely in multiple ways) within the LSST dataset.
\item Creation of methods to generate galaxy environmental measures as Level 3 data products for use by the entire project. 
\end{enumerate}
}
\end{task}

\subsection{Enabling and Optimizing Measurements of Galaxy Clustering}
\tasktitle{Enabling and Optimizing Measurements of Galaxy Clustering}
\begin{task}
\label{task:clss:clustering}
\motivation{
Contemporary galaxy surveys have
transformed the study of large-scale structure, enabling high
precision measurements of clustering statistics. The correlation
function provides a fundamental way to characterize the galaxy
distribution. The dependence of clustering on galaxy properties and
the evolution of clustering provide fundamental constraints on
theories of galaxy formation and evolution. Interpreting these
measurements provides crucial insight into the relation between
galaxies and dark matter halos. Understanding how galaxies relate to
the underlying dark matter is essential for optimally utilizing
the large-scale distribution of galaxies as a cosmological probe.
}
~\\
\activities{
Support work to define and characterize
the upcoming galaxy samples from LSST to enable clustering
measurements. Several distinct sets of information need to
be made available or be calculable from pipeline data. Such
requirements include a detailed understanding of any selection effects
impacting the observed galaxies, the angular and radial completeness
of the samples, and the detailed geometry of the survey (typically
provided in terms of random catalogs that cover the full survey area).
Further efforts will concern the development, testing, and optimization of
algorithms for measuring galaxy clustering using LSST data. One aspect
to address is how best to handle the large data sets involved
(e.g.~the LSST ``gold'' galaxy sample will include about $4$~billion
galaxies over $18$,$000$~square degrees). Another issue is the development
of a
methodology to incorporate optimally the LSST photo-$z$ estimates with
the angular data to obtain ``2.5-dimensions'' for pristine clustering
measurements.
\\
These algorithms will be tested on realistic LSST mock catalogs, which
will also later serve as a tool for obtaining error estimates on the
measurements.
This endeavor overlaps with DESC-LSS working group efforts, and
requires cooperation of the DESC-PhotoZ working group and the Galaxies
Theory and Mock Catalogs working group.}
~\\
\deliverables{%Deliverables over the next several years from the activities described above include the following:
~
\begin{enumerate}
\item Guidance for LSST galaxy pipelines to include all the necessary
information for measurements of the correlation function and related
statistics to take place once LSST data is available.
\item Development and refinement of techniques for measuring galaxy clustering of
large LSST galaxy samples. Together these will enable realizing the
full potential of LSST data for large-scale structure studies and
galaxy formation inferences thereof.
\end{enumerate}
}
\end{task}


\subsection{Disentangling Complicated Lines-of-Sight}
\tasktitle{Disentangling Complicated Lines-of-Sight}
\begin{task}
\label{task:clss:los}
\motivation{
Lines-of-sight through galaxy clusters and groups rank among the most
challenging lines-of-sight along which to measure reliable photometric
redshifts as crowding of galaxies complicates the basic process
of galaxy photometry, and the presence of significant correlated
LSS complicates interpretation of any galaxy redshift distribution
computed by algorithms that ignore the
presence of LSS.  Numerous science goals require the most robust
probabilistic statements possible as to the location of galaxies along
lines of sight through clusters, for example, identification of
background galaxies for weak-lensing, identification of faint cluster
members to study the evolution of the luminosity function in clusters,
identification of star-forming galaxies in clusters and their infall
regions to probe the physics of quenching of star formation.
}
~\\
\activities{
LSST will deliver the most information rich dataset ever in
relation to the masses and internal structures of clusters and their
infall regions. Moreover, the dataset can be enhanced significantly
via the addition of data at other wavelengths, including X-ray,
millimeter, and near-infrared.
~\\
A tool is therefore envisaged that can take an input catalogue of
cluster centers that has been obtained from LSST or any other dataset
(e.g.~\emph{Planck}, \emph{eROSITA}), extract the
basic L2 LSST photometry of objects within a cone centred on the
cluster center, and compute the redshift probability distribution
$p(z)$ of each galaxy based on a
cluster-specific algorithm. When available, t
his algorithm will take account of the brightness and extent of X-ray
emission, over-density of galaxies as a function of magnitude and
colour, any available spectroscopic redshifts, and the amplitude and extent of
any SZ decrement/increment. The algorithm will likely adopt a Bayesian
hierarchical modeling approach to forward model the problem,
and can be tested on existing datasets from surveys such as the
Local Cluster Substructure Survey (LoCuSS), XXL, and HSC data processed by
DM Stack within LSST.
\\
This work links with efforts on deblending/ICL, forward modeling
of cluster and groups, environmental measures, cluster detection,
complementary data, and work in the DESC Clusters Working Group via 
the determination of $p(z)$ for background galaxies.}
~\\
\deliverables{%Deliverables over the next several years from the activities described above include the following:
~
\begin{enumerate}
\item A new cluster-specific photometric redshift algorithm that can be
applied to a list of cluster detections based on LSST
or external data.
\end{enumerate}
}
\end{task}



\subsection{Forward Modeling of LSST Clusters and Groups}
\tasktitle{Forward Modeling of LSST Clusters and Groups}
\begin{task}
\label{task:clss:cluster_fm}
\motivation{
Most of the interesting cluster and group physics from LSST and its
union with complementary surveys will be derived from studies that
explore the full range of halo mass relevant to groups and clusters
$M_{200}\simeq10^{13}-10^{15}M_\odot$.  This mass range extends
beyond that used by cluster cosmologists (e.g. colleagues in DESC), 
who typically restrict attention to objects with masses 
$M_{200}>10^{14}M_\odot$.
However, the
requirement on controlling systematic biases for cluster/group physics
is roughly an order of
magnitude less stringent than for dark energy science.  Arguably, $\sim10\%$
control of systematic biases in weak-lensing measurements of low
redshift clusters ($\gtrsim2\times10^{14}M_\odot$) has already been
achieved \citep{okabe2013a,applegate2014a,hoekstra2015a,okabe2016a}.
This science task focuses on the challenge of maintaining that level of
control down to smaller masses and out to higher redshifts.
}
~\\
\activities{
A growing number of studies are adopting an approach of forward
modeling the cluster population simultaneously with the cosmological
model to obtain constraints on scaling relations and cosmological
parameters. Here, the idea is to borrow this same approach, but adopt
a fixed cosmological model, broaden the mass range of systems
considered, and expand the forward modeling to include additional
relationships of interest.  For example, one could simultaneously fit
density profile models to the shear profiles, the mass-concentration
relation, and the star-formation rates of clusters and groups.
Overall, this effort will provide a robust Bayesian inference code with which
to constrain the physics of galaxies and hot gas in groups and
clusters, tied directly to the halo mass function via weak-lensing.
Related activities include the selection of elements of the cluster population to include in the model,
and the development and testing of the code on simulated data.
Once developed, the code can be improved by extending the range of physics 
explored by adding more physical relations, and by testing the code on existing 
datasets from pointed surveys (e.g. LoCuSS, others) and wide area surveys (e.g. LoCuSS, DES, others).
}
~\\
\deliverables{%Deliverables over the next several years from the activities described above include the following:
~
\begin{enumerate}
\item A Bayesian inference code to simultaneously model cluster shear
profiles and scaling relations across the full range of halo mass of groups and clusters,
to $\sim10\%$ control on systematics.
\end{enumerate}
}
\end{task}



\end{tasklist}
}
