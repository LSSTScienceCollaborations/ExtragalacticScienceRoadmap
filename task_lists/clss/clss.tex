\section{Clusters and Large Scale Structure}\label{sec:tasks:clss}  
{\justify
The cosmological process of galaxy formation inextricably links
the together, environment and large scale structure with the detailed
properties of galaxy populations. The extent of this connection
ranges from the scales of superclusters down to small groups. The
following preparatory science tasks focus on this critical connection
between galaxy formation, clusters, and large scale structure.

\begin{tasklist}{CLSS}
\subsection{Cluster/LSS Sample Emulator}
\tasktitle{Cluster/LSS Sample Emulator}
\begin{task}
\label{task:clss:emulator}
\motivation{
To prepare for galaxies and galaxy group/cluster science with LSST, we
need to know how many galaxies will be detected in a given range of
redshift, brightness, color, etc., and likewise how many groups and
clusters will be detected in given ranges of redshift, richness, mass,
and other physical parameters.
}
~\\
\activities{
LSST has advanced simulations of its 10-year Wide Fast Deep survey
available from the Operations Simulator. The output databases can be
analyzed to determine the depth LSST is expected to reach in its final
detection image at each sky location, and \citet{awan2016a} turns these depths
into predicted numbers of galaxies as a function of redshift and
brightness.
\\
To predict galaxy sample sizes as a function of physical parameters,
the ``raw'' predicted galaxy numbers from \citet{awan2016a} will be
interfaced with semi-analytical models painted on large N-body
simulations by Risa Wechsler and collaborators. This will extend the
predictions to include observed properties of color, size, morphology
and physical properties of halo mass, stellar mass, and star formation
rate.
\\
To predict group/cluster sample sizes as a function of physical
parameters, the properties such as temperature, richness, etc., will
be painted on to dark matter halos drawn from a numerical simulation.
The properties will be based on simple scaling laws, with the user
allowed freedom to choose the parameters of the scaling laws,
including how they evolve. This will then be interfaced with the
``raw'' predicted galaxy numbers from \citet{awan2016a}, to determine
which of the groups and clusters should be detectable in the LSST
data.}
~\\
\deliverables{Deliverables over the next several years from the activities described above include the following:
\begin{enumerate}
\item Create a public LSST Extragalactic Sample Emulator with a
simple GUI. Enable user input of a range of redshift, and
physical parameters (e.g.~galaxy magnitudes, colors, size, morphology,
cluster richness, mass, temperature, etc.) to estimate the size of a
given sample detected by LSST.
\end{enumerate}
}
\end{task}


\subsection{Identifying and Characterizing Clusters}
\tasktitle{Identifying and Characterizing Clusters}
\begin{task}
\label{task:clss:clusters}
\motivation{
LSST photometry will make it possible to
search for and study the galaxy populations of distant clusters and
proto-clusters over huge volumes of the high-$z$ Universe. These
clusters are testbeds for cosmology, hierarchical structure formation,
intergalactic medium heating and metal enrichment, as well as
laboratories for studying galaxy evolution.
However, standard approaches for identifying clusters, such as the red
sequence method, will be hampered by the limited wavelength coverage
of LSST. 
For example, at $z \gtrsim 1.5$, near-IR photometry is required to 
identify systems with Balmer/$4000$\AA\ breaks. 
To maximize cluster science with LSST, we must devise new techniques
for cluster identification as well as incorporate complementary data
from projects such as \emph{Euclid}, \emph{eROSITA}, etc.
}
~\\
\activities{
Using existing imaging datasets and simulations, algorithms need to be
developed and optimised to identify clusters at intermediate
and high redshift within the LSST footprint. 
Specifically, this work should characterize the selection
function, completeness, and contamination rate for different cluster
identification algorithms. 
This requires realistic light-cone simulations spanning extremely
large volumes, so as to capture significant numbers ($\gg10,000$) of
simulated galaxy clusters at high $z$.
Potential algorithms to be tested include adaptations of RedMaPPer
\citep{rykoff2014a} as well as methods that search for galaxy
overdensities over a range of scales \citep[e.g.,][]{chiang2014a,wang2016a}.
In parallel, a comprehensive search for multiwavelength data
(specifically IR and X-ray imaging) is needed to aid in the search for
high-$z$ clusters and in the confirmation and characterization of
systems at all redshifts.}
~\\
\deliverables{Deliverables over the next several years from the activities described above include the following:
\begin{enumerate}
\item The primary product of this analysis will be improved cluster
identification algorithms that can be applied to LSST data once
science operations commence.
\item In addition, this work will produce a compilation of ancillary data
that will be helpful in cluster identification and characterization,
such as X-ray (e.g.~XCS, eROSITA, etc.), SZ (Planck, SPT, ACT) and
radio (SKA and its pathfinders, SUMSS), within the LSST footprint.
\end{enumerate}
}
\end{task}

\subsection{Developing and Optimizing Measurements of Galaxy Environment}
\tasktitle{Developing and Optimizing Measurements of Galaxy Environment}
\begin{task}
\label{task:clss:environment}
\motivation{
Over the past decade, many studies have
shown that ``environment'' plays a important role in shaping galaxy
properties. For example, satellite galaxies in the local Universe
exhibit lower star formation rates, more bulge-dominated morphologies,
as well as older and more metal-rich stellar populations when compared
to isolated (or ``field'') systems of equivalent stellar mass 
\citep{baldry2006a,cooper2010a,pasquali2010a}.
Unlike spectroscopic surveys, LSST will lack the precise line-of-sight
velocity measurements to robustly identify satellite galaxies in
lower-mass groups, where the expected photo-$z$ precision will greatly
exceed the velocity dispersion of the host halo.
Instead, LSST will likely be better suited to measuring environment by
tracing the local galaxy density (and identifying filaments). However,
LSST is unlike any previous photometric survey and may require new
approaches to measuring environment.
The challenge remains to find the measure(s) of local galaxy density
with the greatest sensitivity to the true underlying density field (or
to host halo mass, etc.), so as to enable analyses of environment's
role in galaxy evolution with LSST.
}
~\\
\activities{
Using mock galaxy catalogs created via
semi-analytic techniques, we will compare different tracers of local
galaxy density (i.e.~''environment'') measured on mock LSST
photometric samples to the underlying real-space density of galaxies
(or to host halo mass). In addition to testing existing density
measures, such as $N^{\rm th}$-nearest-neighbor distance and counts in
a fixed aperture, we will explore new measures that may be better
suited to LSST. For each measure, we will examine the impact of
increasing survey depth and photo-$z$ precision over the course of the
survey.}
~\\
\deliverables{Deliverables over the next several years from the activities described above include the following:
\begin{enumerate}
\item With an improved understanding of the
strengths and weaknesses of different environment measures as applied
to LSST, this effort will yield code to measure local galaxy density
(likely in multiple ways) within the LSST dataset.
\item Create a Level 3 data product for use by the entire project. 
\end{enumerate}
}
\end{task}

\subsection{Enabling and Optimizing Measurements of Galaxy Clustering}
\tasktitle{Enabling and Optimizing Measurements of Galaxy Clustering}
\begin{task}
\label{task:clss:clustering}
\motivation{
Contemporary galaxy surveys have
transformed the study of large-scale structure, enabling high
precision measurements of clustering statistics. The correlation
function provides the most fundamental way to characterize the galaxy
distribution. The dependence of clustering on galaxy properties and
the evolution of clustering provide fundamental constraints on
theories of galaxy formation and evolution. Interpreting these
measurements provides crucial insight into the relation between
galaxies and dark matter halos. Understanding how galaxies relate to
the underlying dark matter is also essential for optimally utilizing
the large-scale distribution of galaxies as a cosmological probe.
}
~\\
\activities{
Preparatory work will be along two main
tracks. The first one will be support work to define and characterize
the upcoming galaxy samples from LSST to enable clustering
measurements from them. Several distinct sets of information need to
be made available or be calculable from pipeline data. Such
requirements include a detailed understanding of any selection effects
impacting the observed galaxies, the angular and radial completeness
of the samples, and the detailed geometry of the survey (typically
provided in terms of random catalogs that cover the full survey area).
\\
The second track will be the development, testing, and optimization of
algorithms for measuring galaxy clustering using LSST data. One aspect
to address is how best to handle the large data sets involved
(e.g.~the ``gold'' galaxy sample will include about $4$~billion
galaxies over $20$,$000$~square degrees). Another is to develop the
methodology to optimally incorporate the LSST photo-$z$ estimates with
the angular data to obtain ``2.5-dimensions'' for pristine clustering
measurements.
\\
These algorithms will be tested on realistic LSST mock catalogs, which
will also later serve as a tool for obtaining error estimates on the
measurements.
This endeavor overlaps with DESC-LSS working group efforts, and
requires cooperation of the DESC-PhotoZ working group and the Galaxies
Theory and Mock Catalogs working group.}
~\\
\deliverables{Deliverables over the next several years from the activities described above include the following:
\begin{enumerate}
\item Ensuring LSST galaxy pipelines include all the necessary
information for measurements of the correlation function and related
statistics to take place once data is available
\item Developing and refining techniques for measuring galaxy clustering of
large LSST galaxy samples. Together these will enable realizing the
full potential of LSST data for large-scale structure studies and
galaxy formation inferences thereof.
\end{enumerate}
}
\end{task}


\subsection{Disentangling Complicated Lines of Sight}
\tasktitle{Disentangling Complicated Lines of Sight}
\begin{task}
\label{task:clss:los}
\motivation{
Lines of sight through galaxy clusters and groups are the most
challenging lines of sight along which to measure reliable photometric
redshifts because crowding of galaxies complicates the basic process
of galaxy photometry, and the presence of significant correlated
large-scale structure (LSS) complicates interpretation of the P(z) of
the galaxies that has been computed by an algorithm that ignores the
presence of the LSS.  Numerous science goals require the most robust
probabilistic statements possible as to the location of galaxies along
lines of sight through clusters, for example, identification of
background galaxies for weak-lensing, identification of faint cluster
members to study the evolution of the luminosity function in clusters,
identification of star-forming galaxies in clusters and their infall
regions to probe the physics of quenching of star formation.
}
~\\
\activities{
The LSST will deliver the most information rich dataset ever in
relation to the masses and internal structures of clusters and their
infall regions. Moreover, the dataset can be enhanced significantly
via the addition of data at other wavelengths, including X-ray,
millimeter, and near-infrared.
\\
A tool is therefore envisaged, that can take an input catalogue of
cluster centres that has been obtained from LSST or any other dataset
(e.g.~\emph{Planck}, \emph{eROSITA}).  The tool will pull out the
basic L2 LSST photometry of objects within a cone centred on the
cluster centre, and compute the $p(z)$ of each galaxy based on a
cluster-specific algorithm. This algorithm will take account of the
following where they are available: brightness and extent of X-ray
emission, over-density of galaxies as a function of magnitude and
colour, any available spectroscopic redshifts, amplitude and extent of
any SZ decrement/increment. The algorithm will likely adopt a Bayesian
hierarchical modelling approach to forward model the problem.  The
algorithm can be tested on existing datasets from surveys such as the
Local Cluster Substructure Survey (LoCuSS), XXL, HSC data processed by
DM Stack within LSST, and any others that would like to join in.
\\
This work has links with the work on deblending/ICL, forward modelling
of cluster and groups, environmental measures, cluster detection,
complementary data, and also work in the DESC Clusters WG via $p(z)$
of background galaxies.}
~\\
\deliverables{Deliverables over the next several years from the activities described above include the following:
\begin{enumerate}
\item A new cluster-specific photometric redshift algorithm that can be
applied to a list of cluster detections that is itself based on LSST
or external data.
\end{enumerate}
}
\end{task}



\subsection{Forward Modeling of LSST Clusters and Groups}
\tasktitle{Forward Modeling of LSST Clusters and Groups}
\begin{task}
\label{task:clss:cluster_fm}
\motivation{
Most of the interesting cluster and group physics from LSST and its
union with complementary surveys will be derived from studies that
explore the full range of halo mass relevant to groups and clusters:
$M_{200}\simeq10^{13}-10^{15}M_\odot$.  This is a wider range than
cluster cosmologists (e.g. colleagues in DESC, with whom we
collaborate) aim to incorporate into their cosmological inference --
they restrict attention to $M_{200}>10^{14}M_\odot$.
\\
Another important difference between the cluster/group physics
explored here, and the dark energy-motivated DESC work, is that the
requirement on controlling systematic biases is roughly an order of
magnitude less stringent here than in DESC.  Arguably, $\sim10\%$
control of systematic biases in weak-lensing measurements of low
redshift clusters ($\gtrsim2\times10^{14}M_\odot$) has already been
achieved \citep{okabe2013a,applegate2014a,hoekstra2015a,okabe2016a}.
Therefore in this Science
Collaboration we have the challenge of maintaining that level of
control down to smaller masses and out to higher redshifts.
\\
A growing number of studies are adopting an approach of forward
modeling the cluster population simultaneously with the cosmological
model to obtain constraints on scaling relations and cosmological
parameters.  Here, the idea is to borrow this same approach, but adopt
a fixed cosmological model, broaden the mass range of systems
considered, and expand the forward modeling to include additional
relationships of interest.  For example, simultaneously fitting
density profile models to the shear profiles, the mass-concentration
relation, and the star-formation rates of clusters and groups.
Overall, this will provide a robust Bayesian inference code with which
to constrain the physics of galaxies and hot gas in groups and
clusters, tied directly to the halo mass function via weak-lensing.
}
~\\
\activities{
Key activities include:
%
\begin{itemize}
\item Select the elements of the cluster population to include in the model
\item Write the first version of the code, and test on simulated (toy model and n-body) data
\item Improve code and consider extending range of physics explored by adding more relations
\item Test code on existing datasets from pointed surveys (e.g. LoCuSS, others) and wide area surveys (e.g. LoCuSS, DES, others)
\item Combine this development work other work packages within Galaxies and DESC
\end{itemize}
}
~\\
\deliverables{Deliverables over the next several years from the activities described above include the following:
\begin{enumerate}
\item Bayesian inference code to simultaneously model cluster shear
profiles, scaling relations (including and beyond cosmological scaling
relations, across the full range of halo mass of groups and clusters,
to $\sim10\%$ control on systematics.
\end{enumerate}
}
\end{task}



\end{tasklist}
}
