\section{High-Redshift Galaxies}\label{sec:tasks:high_z}  

Summary

\begin{tasklist}{HZ}
\tasktitle{Optimizing Galaxy Photometry for High-Redshift Sources}
\begin{task}
\label{task:high_z:photometry}
\motivation{
The identification and study of high-redshift galaxies with LSST hinges on reliable, accurate and optimal measurements of the galaxy flux in all LSST passbands.
Galaxies at redshifts above 7 will only be detected in the LSST y-band and will be non-detections or ``drop-outs'' in the other LSST filters. Galaxies at redshifts above 8 will not be detected at all in the LSST filters but combining LSST with infrared surveys such as Euclid and WFIRST would enable this population to be identified. It is particularly important to have robust flux measurements and robust flux limits for the undetected high-redshift galaxies in the blue LSST filters so this information can be utilized in the high-redshift galaxy selection. Since Euclid and WFIRST are space-based missions with very different spatial resolutions and point spread functions (PSFs) compared to LSST, algorithms also need to be devised to provide homogenous flux measurements for sources across the different surveys.
\\
It is not clear if the current Level 2 data products package will meet all the requirements for high-redshift science with LSST and this therefore needs to be investigated before the start of the survey.
}
\activities{
Firstly, we need to get a clearer picture of what constitutes the LSST Level 2 data products so we can assess whether these will be adequate for the high-redshift science. Issues that we need to understand are: 1) Will photometric catalogues be produced using the reddest LSST (e.g. y-band) images as the detection image? This is critical for high-redshift science as high-redshift galaxies will not be detected in the bluer bands. 2) When computing model galaxy fluxes, will negative fluxes be stored? Negative fluxes for undetected galaxies together with their corresponding errorbars, provide useful input into spectral energy distribution (SED) fitting codes for high-redshift galaxy selection.
 \\
The second major activity will be determining the best approach to combining LSST data with infrared data from Euclid/WFIRST for high-redshift galaxy selection. We will need to determine the optimal measure of an optical-IR colour for sources from these two datasets. There is the additional complication that sources that are resolved in the Euclid/WFIRST data could be blended in LSST and will therefore need to be accurately de-blended, perhaps using the high-resolution IR data as a prior, before a reliable flux and colour measurement can be made. Tests can be run using existing datasets e.g. from the Dark Energy Survey (DES) and HST.}
\deliverables{Deliverables over the next several years from the activities described above include the following:
\begin{enumerate}
\item Determine what constitutes LSST Level 2 data products and document what additional data products will be required for high-redshift science.
\item Develop tools to produce optimal combined photometry from ground and space-based surveys and test these on existing datasets.
\end{enumerate}
}
\end{task}

\tasktitle{High-Redshift Galaxies and Interlopers in LSST Simulations}
\begin{task}
\label{task:high_z:interlopers}
\motivation{
 Before the start of LSST operations, it is important that we are able to test our selection methods for high-redshift galaxies on high-fidelity simulations. Given the wide-field coverage of LSST, it will be uniquely positioned to uncover large samples of the most luminous and massive high-redshift galaxies at the Epoch of Reionisation and beyond. The most significant obstacle to selecting clean samples of such sources from the photometric data, is the presence of significant populations of interlopers e.g. cool stars in our own Milky Way and low-redshift, dusty and/or red galaxies, both of which can mimic the colours of high-redshift sources. Using the LSST simulations, we want to be able to devise the most effective way of separating these different populations, and utilising both photometric and morphological information for the sources. Based on experience with ground-based surveys such as the Dark Energy Survey and VISTA infrared surveys, we expect at least some of the most luminous z > 6 galaxies to be spatially resolved in the LSST images.
}
\activities{
Liaise with the LSST simulations working group to ensure that high-redshift galaxies have been incorporated into the simulations with a representative set of physical properties (e.g. star formation histories, UV-slopes, emission line equivalent widths, dust extinction, metallicity). It is also important that the high-redshift galaxies have the correct number density and size distribution in the simulations. The latter will allow us to investigate how effectively we can use morphology to separate these galaxies from interlopers.
\\
In addition to the high-redshift galaxies, it is equally important from a high-redshift science perspective, that interlopers have been incorporated into the simulations with the correct number densities and colours. Interlopers of particular relevance to the high-redshift searches will be cool stars in our own Milky Way (e.g. L and T-dwarf stars) as well as populations of very red, massive and/or dusty galaxies at lower redshifts of $z\sim2$.
\\
Finally, we may want to consider whether to include colour information in the infrared filters (e.g. those from Euclid/WFIRST) in the simulations as this information will undoubtedly help with the high-redshift selection.
 }
\deliverables{Deliverables over the next several years from the activities described above include the following:
\begin{enumerate}
\item Incorporate high-z galaxies into LSST simulations with a realistic and representative set of properties.
\item Incorporate brown dwarfs into LSST simulations
\item Extend simulations to other datasets beyond LSST (e.g. Euclid/WFIRST filters).
\end{enumerate}
}
\end{task}
\end{tasklist}
