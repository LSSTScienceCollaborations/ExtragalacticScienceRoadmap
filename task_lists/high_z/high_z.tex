\section{High-Redshift Galaxies}\label{sec:tasks:high_z}  

{\justify
Observations of distant galaxies provide critical information
about the efficiency of the galaxy formation process, the end
of the reionization era, the early enrichment of the intergalactic
medium, and the initial conditions for the formation of modern
galaxies at later times. Through its wide area and sensitivity
in $zy$, LSST will probe galaxies out to $z\sim7$ and 
yet further in conjunction with future wide-area infrared surveys.
The following science tasks address outstanding preparatory work
for maximizing high-redshift science with LSST.

\begin{tasklist}{HZ}

\subsection{Optimizing Galaxy Photometry for High-Redshift Sources}
\tasktitle{Optimizing Galaxy Photometry for High-Redshift Sources}
\begin{task}
\label{task:high_z:photometry}
\motivation{
The identification and study of high-redshift galaxies with LSST hinges on reliable, accurate and optimal measurements of the galaxy flux in all LSST passbands.
Galaxies at redshifts above $z\sim7$ will only be detected in the LSST $y$-band and will be non-detections or ``drop-outs'' in the other LSST filters.
Galaxies at redshifts $z>8$ will not be detected at all in the LSST filters,
but combining LSST with infrared surveys such as Euclid and WFIRST would enable the identification
of this population. 
Robust flux measurements or limits for the undetected high-redshift galaxies in the blue LSST filters will prove particularly important, as this information enables efficient
high-redshift galaxy selection.
The highest redshift searches will LSST will necessarily require combining with space-based
infrared surveys like Euclid and WFIRST.
Since Euclid and WFIRST will provide data with very different spatial resolutions and point spread functions (PSFs) compared to LSST, algorithms also need to be devised to provide homogeneous flux measurements for sources across the different surveys.
It remains unclear whether the current LSST Level 2 data will meet all the requirements for identifying and characterizing high-redshift galaxy populations, 
motivating an investigation before the start of LSST operations.
}
~\\
\activities{
First, the potential need for Level 3 data products beyond the baseline LSST Level 2 catalog
requires clarification. Photometric catalogues produced using the reddest LSST (e.g. $z$- or $y$-band) images as the detection image will prove critical for high-redshift science as high-redshift galaxies will not be detected in the bluer bands. Similarly, negative fluxes for undetected galaxies together with their corresponding errorbars provide useful input into spectral energy distribution (SED) fitting codes for high-redshift galaxy selection. Coordinating with the LSST Project to ensure
the application of forced photometry in Level 2 in a manner appropriate for high-redshift galaxy
selection may be sufficient, or Level 3 data products may be required.\\
~\\
Second, the determination of a suitable approach to combining LSST data with infrared data from Euclid/WFIRST for high-redshift galaxy selection will be required, including optimal measures of an optical-IR color for sources from these combined datasets. 
Sources resolved in Euclid or WFIRST data could be blended in LSST, and may therefore
require deblending using the higher resolution IR data as a prior before reliable flux and
color measurements can be made. The engineering of this combined analysis likely will require
Level 3 efforts, and tests using existing datasets (e.g., Dark Energy Survey and Hubble
Space Telescope data) may already commence.}
~\\
\deliverables{%Deliverables over the next several years from the activities described above include the following:
~
\begin{enumerate}
\item Clarification of LSST Level 2 data suitability for high-redshift science, and an identification of any Level 3 needs.
\item Development of Level 3 tools to produce optimal combined photometry from ground and space-based surveys, and the testing of these tools on existing datasets.
\end{enumerate}
}
\end{task}

\subsection{High-Redshift Galaxies and Interlopers in LSST Simulations}
\tasktitle{High-Redshift Galaxies and Interlopers in LSST Simulations}
\begin{task}
\label{task:high_z:interlopers}
\motivation{
Before the start of LSST operations, the testing of selection methods for high-redshift
galaxies on high-fidelity simulations will provide essential validation of the utility of
LSST data for studying distant galaxy populations.
Given its wide-field coverage, LSST will uniquely uncover large samples of the most luminous and potentially massive high-redshift galaxies at the Epoch of Reionization \citep{robertson2007a}.
The most significant obstacle to selecting clean samples of such sources from the photometric data
is the presence of significant populations of interlopers, such as 
cool brown dwarfs in our own Milky Way and low-redshift, dusty and/or red galaxies.
These objects can mimic the colors of high-redshift sources and therefore
prove difficult to distinguish.
This issue is particularly a problem for the highest redshift objects detected by LSST, which, unless data at redder wavelengths is available (such as near-infrared imaging from VISTA/Euclid/WFIRST, and/or mid-infrared imaging from Spitzer/WISE) will be only detected in one or two red-optical filters.
Using the LSST simulations, one wants to devise the most effective way of separating these different populations by utilizing both photometric and morphological information for the sources.
Based on experience with ground-based surveys such as the Dark Energy Survey and VISTA, 
one expects LSST images to spatially resolve at least some of the most luminous $z > 6$ galaxies \citep{willott2013a, bowler2017a}.
For fainter high-redshift galaxies however, a morphological distinction between faint ultra-cool brown dwarfs may not be possible, and further information such as near-infrared colours or proper motions will be required for identification.
}
~\\
\activities{
Liaise with the LSST Project simulations working group to ensure that high-redshift galaxies have been incorporated into the simulations with a representative set of physical properties (e.g., star formation histories, UV-slopes, emission line equivalent widths, dust extinction, and
metallicity). Ensure that high-redshift galaxies have the correct number density and size distribution in the simulations, allowing for investigations to characterize
how effectively morphology can separate high-redshift galaxies from low-redshift interlopers.
The high-redshift quasar population should also be included, as these have comparable number densities to the brightest galaxies at these redshifts and are typically indistinguishable with broad-band photometry only.
Incorporate interloper populations into the simulations with the correct number densities and colors, including cool Milky Way stars
(e.g., M, L and T-brown dwarfs) as well as populations of very red, massive,
and/or dusty galaxies at lower redshifts of $z\sim2$.
Determine the degree to which LSST selection of high-redshift galaxies effectively requires 
color information from infrared filters provided by external surveys
(i.e., Euclid or WFIRST). 
 }
 ~\\
\deliverables{%Deliverables over the next several years from the activities described above include the following:
~
\begin{enumerate}
\item Incorporation of high-$z$ galaxies and quasars into LSST simulations with a realistic and representative set of properties.
\item Incorporation pf cool Milky Way brown dwarfs into LSST simulations.
\item Predictions of the likely number density of brown dwarfs over the different DDFs.
\item Extension of simulations to other datasets beyond LSST (e.g., Euclid and WFIRST filters).
\end{enumerate}
}
\end{task}
\end{tasklist}
}
