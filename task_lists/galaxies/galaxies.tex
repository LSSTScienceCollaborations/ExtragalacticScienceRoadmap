% LSST Extragalactic Roadmap
% Chapter: task_lists 
% Section: galaxies
% First draft by 

\section{Galaxy Evolution Task Lists}\label{sec:tasks:gal:intro}  

The LSST design, and to a certain extent the design of the data-management
system, is optimized to carry out the core science mission. For measurements
of dark-energy, that generally means treating galaxies as ``tracer particles'' --
using statistical measures of ellipticity and position provide statistical
constraints on large-scale structure and cosmic geometry. While many of the
DESC tasks are directly relevant to studying galaxy evolution, they are 
incomplete. In particular, studies of galaxy evolution require more attention to 
optimizing multi-wavelength supporting data, different kinds of spectroscopy, different
kinds of simulations and theoretical support, and greater attention to detection
and characterization of low-surface brightness features or unusual morphologies.

The task list presented here highlights the preparation work needed in the next 3-4
years. Of primary importance are tasks that might influence the detailed survey
design or the algorithms used in the DM to construct catalogs. These are the most
urgent. Also included are activities that can be reasonably independent of the
LSST survey design and DM optimization, but which will ensure good support for
LSST galaxy studies.

\begin{tasklist}{G}
\tasktitle{Example Task List}
\begin{task}
\label{task:label_for_this_task}
\motivation{Put Science Motivation Here}
\activities{Described Activities Here}
\deliverables{
	Deliverables over the next several years from the activities described above include the following:
	\begin{enumerate}
		\item a deliverable
		\item another deliverable
	\end{enumerate}
}
\end{task}
\end{tasklist}  

\subsection{Techniques, Algorithms, or Software Development} \label{sec:tasks:gal:techniques_and_algorithms}

\begin{tasklist}{G-TAS}
\tasktitle{Techniques for finding low-surface-brightness features or galaxies}
% Tidal streams
% Intracluster diffuse light
\begin{task}
\label{task:gal:lsb}
\motivation{
A huge benefit of LSST relative to prior large-area surveys will be its ability to detect 
low-surface-brightness (LSB) features associated with galaxies. This includes tidal streams and
other features associated with past and ongoing mergers, it includes intra-cluster and
intra-group light, and it includes relatively nearby, extended low-surface-brightness galaxies. 
Prior to LSST, typical studies of the low-surface-brightness universe have focused
on relatively small samples, often selected by criteria that are difficult to quantify
or reproduce in theoretical models. Measurements of the LSB features themselves themselves
are challenging, often requiring hand-tuning and interactive scientific judgment. This is
important for accurately quantifying what we observe, but such interactive tuning
of the measurements (a) is not something that can be applied on the LSST scale and (b) 
is difficult to apply to theoretical models. For LSST it is crucial that we automate 
the detection and characterization of LSB features, at least to the point where samples
for further study can be selected via database queries, and where the completeness of
samples returned from such queries can be quantified.
}

\activities{
Several activities are of crucial importance: (1) simulating realistic LSB features, (2) 
using the simulations to optimize detection and measurement, (3) ensuring that LSST
level-2 processing strategies and observing strategies are at least cognizant
of needs of LSB science and (4) developing a strategy for finding and measuring LSB features through
some combination of level 2 measurements, database queries, and level 3 processing.\\
It is important to insert realistic low-surface-brightness
features into LSST simulated images and try to extract and measure them, exploring
different techniques or algorithms for doing the detection and measurement. Because the LSB objects
are sparse on the sky, making realistic LSST sky images is probably not the most efficient
way to accomplish this; more targeted simulations with a higher density of 
LSB objects are needed. The simulated observations need to be realistic in their
treatment of scattered light, particularly scattering from bright stars which
may or may not be in the actual field of view of the telescope. 
Scattering from bright stars is likely to be the primary source of contamination 
when searching for extended LSB features. Ideally, the LSST scattered-light model,
tuned by repeated observations, will be sufficiently good that these contaminants
can be removed or at least flagged at level 2. Defining the metrics for ``sufficiently good,''
based on analysis of simulations, is an important activity that needs early work to 
help inform LSST development. Including Galactic cirrus in the simulations is important
for very large-scale LSB features. Including a cirrus model as part of the LSST background
estimation is worth considering, but it is unclear yet whether the science benefit
can justify the extra effort.  \\
Because the LSST source extraction is primarily
optimized for finding faint, barely-resolved galaxies, it is going to be challenging to
optimize simultaneously for finding large LSB structures and cataloging them as 
one entity in the database. For very large structures, analysis of the LSST ``sky background''
map, might be the most productive approach. We need to work with the LSST project
to make sure the background map is stored in a useful form, and that background
measurements from repeated observations can be combined to separate the fluctuating
foreground and scattered light from the astrophysically interesting signal from extended
LSB structures.  Then, we need strategies for measuring these background maps, characterizing
structures, and developing value-added catalogs to supplement the level 2 database.\\
For smaller structures, it is likely that the database will contain pieces 
of the structure, either as portions of a hierachical
family of deblended objects, or cataloged as separate objects. Therefore, we need to 
develop strategies for querying the database to find such structures and either extract
the appropriate data for customized processing, or develop ways to put back together
the separate entries in the database. A possible value-added catalog, for example, from
the galaxies collaboration might be an extra set of fields for the database to indicate 
which separate objects are probably part of the same physical entity. This would
be sparsely populated in the first year or two of LSST, but by the end of the survey 
could be a useful resource for a wide variety of investigations.
}

\deliverables{
Deliverables over the next several years from the activities described above include the following:
\begin{enumerate}
\item realistic inputs of LSB galaxies or LSB features for the LSST image simulations; 
\item custom simulations;
\item algorithms for finding and measuring LSB features;
\item input to the Project on scattered-light mitigation and modeling strategies; 
\item input to the project on photometric and morphological parameters to measure/store at level 2;
\item query strategies and sample queries for finding LSB structures; and
\item a baseline concept for a value-added database of LSB structures
\end{enumerate}
}
\end{task}

\tasktitle{Techniques for identifying and deblending overlapping galaxies}
\begin{task}
\label{task:gal:deblending}
\motivation{
The Level 2 data products are the most relevant starting point
for galaxy-evolution science. In the LSST nomenclature, {\tt Objects}
represent astrophysical entities (stars, galaxies, quasars, etc.), while
{\tt Sources} represent their single-epoch observations. 
The master list of Objects in Level 2 will be generated by associating
and deblending the list of single-epoch source detections and the
lists of sources detected on coadds. The exact strategies for doing
this are still under active development by the LSST project, and 
engagement with the science community is essential. While each
data release will have unique object IDs, it will be a huge impediment
for LSST science if the first few generations of catalogs turn out
severely the limit the science that can be done via database queries. \\
For galaxies science, the issue of deblending is of critical importance. 
For example, searches for high-redshift galaxies via color selection 
or photometric redshifts involve model or template spectra that make
the prior assumption that the object in question is a single object at
one redshift, not a blend of two objects at two different redshifts.
Therefore to get a reliable estimate of the evolution of classes of galaxies
over redshift, we need to (a) have reasonably clean catalogs to start with
and (b) be able to model the effects of blending on the sample selection
and derivation of redshift and other parameters.  This is critical
not just for galaxy-evolution science, but for lensing and large-scale
structure studies. This is just one example. Another is the evolution
of galaxy morphologies, where the effects of blending and confusion 
may well be the dominant source of uncertainty. \\
The plan for the level-2 catalogs is that sources are hierarchically
deblended and that this hierarchy is maintained in the catalog.
Scientifically important decisions are still to be made about whether
and how to use color information in the deblending, and how to divide
the flux between overlapping components. Even if the Project is doing
the development work, engagement with the community is important for
developing tests and figures of merit to optimize the science return.
}
\activities{
Preparations for LSST in this area involve working both with simulations
and real data. The current LSST image simulations already have realistic source densities,
redshift distributions, sizes, and color distributions. However, the
input galaxies do not have realistic morphologies. At least some simulations
with realistic morphologies are needed, especially for the Deep Drilling Fields. 
Inputs should come both from hydrodynamical simulations (where ``truth'' is known),
{\it Hubble} images, and images from precursor surveys such as CFHTLS, DES and HyperSuprimeCam.
The science collaborations should help provide and vet inputs. \\
More challenging is to come up with techniques and algorithms to improve the
deblending. When two galaxies at different redshifts overlap, using observations
from all the LSST filters and perhaps even EUCLID and WFIRST might 
help to disentangle them. Some attempts have been made over the past few years
to incorporate color information into the deblending algorithm, but this needs
much more attention, not only for developing and testing algorithms, but for
deciding on figures-of-merit for their performance.
}
\deliverables{
Deliverables over the next several years from the activities described above include the following:
\begin{enumerate}
\item providing realistic galaxy image inputs to the ImSim team; 
\item developing tests and figures of merit to quantify the effects on several science objectives;
\item assessing the current baseline plan for level-2 deblending and for parameter estimation for blended objects;  and 
\item developing prototype implementations of deblending algorithms that take advantage of the LSST color information.
\end{enumerate}
}
\end{task}

\tasktitle{Optimizing Galaxy Morphology Measurements}
% techniques for identifying mergers
% Bayesian techniques for inference from large data sets
% Merging human classification and machine learning
\begin{task}
\label{task:gal:morphology}
\motivation{
Measurements of galaxy morphologies are an important tool for constraining models
of galaxy evolution. While fairly simple measures of galaxy ellipticity and position
angles may be sufficient for the Dark Energy science goals, other kinds of
measurements are needed for galaxy-evolution science.  The ``multifit'' approach of 
fitting simple parametric models to galaxy profiles has been the baseline plan. 
This will be useful but insufficient. For well-resolved galaxies it is desirable
to have separate measures of bulge and disk, and spiral-arm structure, measures of 
concentration, asymmetry, and clumpiness. These ought to be measured as part of 
the level 2 processing, to enable database queries to extract subclasses of galaxies. 
Both parametric and non-parametric measures are desirable.
While there will no doubt be optimization in level 3 processing, it is important
to have enough information in the level 2 output products to pick reasonable subsets
of galaxies.
}
\activities{The preparation work, therefore, focuses on defining measures to enable
these queries. Two aspects of LSST data make this a significant research project: 
the fact that LSST provides multi-band data with a high degree of uniformity, and the
fact that the individual observations will have varying point-spread functions.
The former offers the opportunity to use much more information than has been 
generally possible. The latter means that it will take some effort to optimize and 
calibrate the traditional non-parametric measure of morphology (e.g. the CAS, GINI and M20 parameters),
develop new LSST-optimized parameters, and optimize their computation to avoid
taxing the level-2 pipeline.\\
Given the very large data set and the uncertainty in how to use specific morphological
parameters to choose galaxies in certain physical classes (e.g. different merger
stages or stages of disk growth), it is important to have extensive 
training both from hydrodynamical simulations
with dust (where physical truth is known, even if the models are imperfect) and
from observations where kinematics or other information provide a good 
understanding of the physical nature of the object. These training sets ought to 
be classified by humans (still the gold-standard for image classification) and via
machine-learning techniques applied to the morphological measurements. A series
of ``classification challenges'' prior to the LSST survey could help to refine the techniques. 
}
\deliverables{
Deliverables over the next several years from the activities described above include the following:
\begin{enumerate}
\item providing realistic galaxy image inputs for classification tests to the ImSim team; 
\item human classification of the images;
\item machine-learning algorithms to be tested and developed into suitable SQL queries;
\item developing a menu of candidate morphological measurements for level 2 and level 3 processing; and
\item developing tests and figures of merit to quantify the effects on several science objectives.
\end{enumerate}
}
\end{task}

\tasktitle{Optimizing Galaxy Photometry} 
% background subtraction
% optimal co-adds
% best flux estimator
% image quality metrics 
% forced photometry with separate central point source (important for AGN)
% Using high-resolution priors where available
\begin{task}
\label{task:gal:photometry}
\motivation{
Systematic uncertainties will dominate over random uncertainties for almost any
research question one can imagine addressing with LSST. The most basic measurement
of a galaxy is its flux in each band, but this is a remarkably subtle measurement
for a variety of reasons: galaxies do not have well-defined edges, their shapes
vary, they have close neighbors, they cluster together, and lensing affects both
their brightness and clustering. These factors all affect photometry in systematic
ways, potentially creating spurious correlations that can obscure or masquerade as
astrophysical effects. For example, efforts to measure the effect of neighbors
on galaxy star-formation rates can be thrown off if the presence of a neighbor
affects the basic photometry. Measurements of galaxy magnification or measurements
of intergalactic dust can be similarly affected by systematic photometric biases.
It is thus important to hone the photometry techniques prior to the survey to 
minimize and characterize the biases. Furthermore, there are science topics that
require not just photometry for the entire galaxy, but well-characterized photometry
for sub-components, such as a central point-source or a central bulge. 
}
\activities{
The core photometry algorithms will end up being applied in level 2 processing, 
so it is important that photometry be vetted for a large number of potential
science projects before finalizing the software. Issues include the following.
(1) Background estimation, which, for example, can greatly affect the photometry
for galaxies in clusters or dwarfs around giant galaxies. (2) Quantifying the
biases of different flux estimators vs. (for example) distances to and fluxes
of their neighbors. (3) Defining optimal strategies to deal with the varying
image quality. (4) Defining a strategy for forced photometry of a central point
source. For time-varying point-sources, the image subtractions will give a
precise center, but will only measure the AC component of the flux. Additional
measurements will be needed to give the static component. (5) Making use
of high-resolution priors from either Euclid or WFIRST, when available. 
Because photometry is so central to much of LSST science, there will need to
be close collaboration between the LSST Project and the community. 
}
\deliverables{
Deliverables over the next several years from the activities described above include the following:
\begin{enumerate}
\item developing metrics for various science cases to help evaluate the level 2 photometry;
\item providing realistic inputs for difficult photometry cases (e.g. close neighbors, clusters of galaxies, AGN in galaxies of various morphologies);
\end{enumerate}
Deliverables over the longer term include develping optimal techniques for forced photometry
using priors from the space missions.
}
\end{task}

\tasktitle{Optimizing Measurements of Stellar Population Parameters} 
% Strategies for dealing with strong covariance of parameter estmates
\begin{task}
\label{task:gal:stellarpops}
\motivation{
The colors of galaxies carry information about their star-formation histories, 
each interval of redshift being a snapshot of star-formation up until that time.
Unfortunately, estimates of star-formation rates and star-formation histories 
for a single galaxy based on only the LSST bands will be highly uncertain, 
due largely to degeneracies between age, dust extinction and metallicity. 
Strategies for overcoming the degeneracies include hierarchical modeling -- using
ensembles of galaxies to constrain the hyper-parameters that govern 
the star-formation histories of sets of galaxies rather than individuals, 
and using ancillary data from other wavelengths. 
}
\activities{
Activities in this area include developing scalable techniques for 
hierarchical Bayesian inference on very large data sets. These can be
tested on semi-analytical or hydrodynamical models, where the answer is known,
even if it does not correctly represent galaxy evolution. The models should
also be analyzed to find simple analytical expressions for star-formation
histories, chemical evolution and the evolution and behavior of dust to
make the Bayesian inference practical.\\
Another important activity is to identify the ancillary data sets and
observing opportunities, especially for the deep fields.
}
\deliverables{
Deliverables over the next several years from the activities described above include the following:
\begin{enumerate}
\item developing and refining techniques for constraining star-formation histories of large ensembles of galaxies;
\item providing model inputs to guide in developing these techniques;
\item refining the science requirements for ancillary multi-wavelength data to support LSST.
\end{enumerate}
}
\end{task}

\tasktitle{Software Integration}
% Level 2 and Level 3 software
\begin{task}
\label{task:gal:integration}
\motivation{
The LSST Project is responsible for level 2 data processing, and the community
is expected to any processing beyond that as level 3. Furthermore, some algorithms developed
as part of the level 3 effort are expected to migrate to level 2. There needs
to be strong coordination between the Project and the community for this concept
to work. This includes training in developing level 3 software and community engagement in 
defining the requirements and interfaces. 
}
\activities{
The most urgent activity is to develop some early prototypes of level 3 software
so that the interfaces can be worked out on realistic use cases.
}
\deliverables{}
\end{task}

\end{tasklist}

\subsection{Precursor Observations or Synergy with Other Facilities} \label{sec:tasks:gal:precursor}

\begin{tasklist}{G-PO}
\tasktitle{Redshift surveys in the Deep Drilling fields}
\begin{task}
\label{task:gal:redshift_surveys_dd_fields}
\motivation{}
\activities{}
\deliverables{}
\end{task}

\tasktitle{Ancillary Data in Deep Drilling fields}
\begin{task}
\label{task:gal:ancillary_data_dd_fields}
\motivation{}
\activities{}
\deliverables{}
\end{task}

\tasktitle{Photometric redshift training and calibration}
% Emphasize differences in requirements relative to DE
\begin{task}
\label{task:gal:photz_calibration}
\motivation{}
\activities{}
\deliverables{}
\end{task}

\tasktitle{Joint use of spectroscopic and photometric redshifts}
\begin{task}
\label{task:gal:spec_plus_phot_redshifts}
\motivation{}
\activities{}
\deliverables{}
\end{task}

\end{tasklist}

\subsection{LSST-targeted theory or simulations} \label{sec:tasks:gal:simulations}

\begin{tasklist}{G-TS}
\tasktitle{Image simulations of galaxies with complex morphologies}
% Mergers
% Tidal features
% Stellar halos
% Vary the galaxy-evolution model
\begin{task}
\label{task:gal:image_simulations}
\motivation{}
\activities{}
\deliverables{}
\end{task}

\tasktitle{Rare objects}
% extreme over/underdensities
% massive early galaxies
% extremely supermassive black holdes
\begin{task}
\label{task:gal:rare_objects}
\motivation{}
\activities{}
\deliverables{}
\end{task}

\tasktitle{Cosmic Variance estimators}
% Develop simple tools...encourage their use
\begin{task}
\label{task:gal:cv_estimators}
\motivation{}
\activities{}
\deliverables{}
\end{task}

\tasktitle{Nearby dwarfs: surface brightness fluctuations}
\begin{task}
\label{task:gal:dwarf_sb_fluctuations}
\motivation{}
\activities{}
\deliverables{}
\end{task}

\tasktitle{Testing group and void finders}
\begin{task}
\label{task:gal:group_finders}
\motivation{}
\activities{}
\deliverables{}
\end{task}

\end{tasklist}

\subsection{Databases and Data Services} \label{sec:tasks:gal:databases}

\begin{tasklist}{G-DDS}
\tasktitle{Data structures to characterize survey biases and completeness}
\begin{task}
\label{task:gal:data_structures}
\motivation{}
\activities{}
\deliverables{}
\end{task}

\tasktitle{Queries to find unusual class of objects}
% mergers
% tidal streams
% nearby dwarf candidates
% morphologically disturbed close pairs
\begin{task}
\label{task:gal:queries}
\motivation{}
\activities{}
\deliverables{}
\end{task}

\tasktitle{Compact representations of likelihood functions}
\begin{task}
\label{task:gal:likelihoods}
\motivation{}
\activities{}
\end{task}

\end{tasklist}
