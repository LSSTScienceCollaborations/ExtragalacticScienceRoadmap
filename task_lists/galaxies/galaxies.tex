% LSST Extragalactic Roadmap
% Chapter: task_lists 
% Section: galaxies

\section{Galaxy Evolution Task Lists}\label{sec:tasks:gal:intro}  
{\justify
The LSST Project optimized the observatory and data management design
to execute successfully and efficiently the core LSST science mission. 
For measurements
of dark energy, that optimization generally means treating galaxies as ``tracer particles'' --
 using statistical measures of ellipticity and position to provide statistical
constraints on large-scale structure and cosmic geometry. While many of the
DESC tasks relate directly to studying galaxy evolution, they remain 
incomplete. In particular, studies of galaxy evolution require more attention to 
optimizing multi-wavelength supporting data, obtaining different kinds of spectroscopy, 
performing different
kinds of simulations and other theoretical support, and a greater attention to 
the detection
and characterization of low-surface brightness features or unusual morphologies.

The task list presented here highlights the preparatory research needed in the next $\sim3-4$
years.
Tasks of primary importance and particular urgency include those that might influence the detailed survey
design or the algorithms used in the LSST Software Stack to construct catalogs. 
Other critical tasks remain reasonably independent of the
LSST survey design and data pipeline optimization, but will help ensure good support for
LSST galaxy studies.

\subsection{Techniques, Algorithms, or Software Development} \label{sec:tasks:gal:techniques_and_algorithms}

\begin{tasklist}{G-TAS}
\tasktitle{Techniques for finding low-surface-brightness features or galaxies}
% Tidal streams
% Intracluster diffuse light
\begin{task}
\label{task:gal:lsb}
\motivation{
Important scientific benefits of the LSST dataset
relative to prior large-area surveys include its ability to detect 
low-surface-brightness (LSB) features associated with galaxies. 
This ability includes the identification of tidal streams and
other features associated with past and ongoing mergers, intra-cluster and
intra-group light, and relatively nearby, extended low-surface-brightness galaxies. 
Prior to LSST, typical studies of the low-surface-brightness universe focused
on relatively small samples, often selected by criteria that prove difficult to quantify
or reproduce in theoretical models. Measurements of the LSB features themselves
can challenge pipelines and subsequent analysis, 
and often require both hand-tuning and interactive scientific judgment. This manual
attention serves to help quantify accurately what we observe, but such
interacting tuning of the measurements does not scale to the LSST dataset and
can prove difficult to apply to theoretical models.
For LSST, we must automate the detection and characterization of LSB features,
at least to the point where we can select samples
for further study via database queries and quantify the completeness or
other statistical properties of
those retrieved samples.
}
~\\
\activities{
Several crucial activities include: (1) simulating realistic LSB features, (2) 
using the simulations to optimize detection and measurement, (3) informing LSST
Level 2 processing and observing strategies about the needs of LSB science, 
and (4) developing a strategy for finding and measuring LSB features through
some combination of Level 2 measurements, database queries, and Level 3 processing.\\
The instertion of realistic LSB features into LSST simulated images will provide
``data challenges'' to test methods for their extraction and measurement, allowing
for the exploration of different techniques or algorithms for performing LSB
feature detection and characterization.
Because the LSB objects sparsely populate the sky,
making realistic LSST sky images will probably prove inefficient.
More targeted simulations with a higher density of LSB object will better
enable the efficient exploration of LSB feature detection and analysis.
Simulated observations must realistically treat scattered light, particularly
scattering from bright stars that may or may not fall in the actual field of
view of the telescope. Scattering from bright stars will likely contribute the
primary source of contamination when searching for extended LSB features.
Ideally, the LSST scattered-light model,
tuned by repeated observations, will perform sufficiently well and enable the removal
or flagging of these contaminants at Level 2.  
Defining the associated performance metrics for
based on analysis of simulation represents an important activity that needs early work to 
help inform LSST development. Including Galactic cirrus in the simulations is important
for very large-scale LSB features. Including a cirrus model as part of the LSST background
estimation is worth considering, but the science benefit gained from the
additional effort remains unclear.\\
Because the LSST source extraction is primarily
optimized for finding faint, barely-resolved galaxies,
simultaneously finding large LSB structures and cataloging them as 
one entity in the LSST database may pose challenges. 
For very large structures, analysis of the LSST ``sky background''
map might constitute the most productive approach. 
We need to work with the LSST Project
to make sure the Software Stack stores the background map in a useful form, and that background
measurements from repeated observations can be combined to separate the fluctuating
foreground and scattered light from the astrophysically interesting signal owing to extended
LSB structures.
Then, we need strategies for measuring these background maps, characterizing
structures, and developing value-added catalogs to supplement the Level 2 database.\\
For smaller structures, the database likely will contain pieces 
of the structure, either as portions of a hierarchical
family of deblended objects or as separate catalog entries. Therefore, we need to 
develop strategies for querying the database to find such structures and either extract
the appropriate data for customized processing, or develop ways to put back together
the separate entries in the database. A possible value-added catalog, for example, from
the Galaxies Science Collaboration might include an extra set of fields for the database to indicate 
which separate objects likely probably originate from the same physical entity. These additional fields would
remain sparsely populated in the first year or two of LSST, but by the end of the survey 
the relational connections between deblended objects may prove a 
useful resource for a wide variety of investigations.
}
~\\
\deliverables{
Deliverables over the next several years from the activities described above include the following:
\begin{enumerate}
\item Creations of realistic inputs of LSB galaxies or LSB features for the LSST image simulations.
\item Development of algorithms for finding and measuring LSB features.
\item Input to the Project on scattered-light mitigation and modeling strategies.
\item Input to the Project on photometric and morphological parameters to measure/store at Level 2.
\item Identification of query strategies and sample queries for finding LSB structures.
\item Engineering of a baseline concept for a value-added database of LSB structures.
\end{enumerate}
}
\end{task}

\subsection{Techniques for Identifying and Deblending Overlapping Galaxies}
\tasktitle{Techniques for Identifying and Deblending Overlapping Galaxies}
\begin{task}
\label{task:gal:deblending}
\motivation{
The Level 2 data products provide the most relevant starting point
for galaxy-evolution science. In the LSST nomenclature, {\tt Objects}
represent astrophysical entities (stars, galaxies, quasars, etc.), while
{\tt Sources} represent their single-epoch observations. 
The Software Stack will generate the master list of Objects in Level 2
by associating
and deblending the list of single-epoch source detections and the
lists of sources detected on co-adds. The exact strategies for 
performing this task still remain under active development by the LSST Project, and 
engagement with the science community will prove essential. While each
data release will provide unique object IDs, if the first few
generations of catalogs limit the science performed through data base
queries the consequences may impede early LSST science.\\
For galaxies science, the issue of deblending holds critical importance. 
For example, searches for high-redshift galaxies via color selection 
or photometric redshifts involve model or template spectra that make
the prior assumption that each analyzed object does not consist of a
blend of two objects at two different redshifts.
Therefore, to get a reliable estimate of the evolution of classes of galaxies
over redshift we need to (a) create reasonably clean catalogs to start with
and (b) model the effects of blending on the sample selection
and derivation of redshift and other parameters.  These issues critically
affect not just galaxy evolution science, but also lensing and large-scale
structure studies. Another example involves the evolution
of galaxy morphologies, where the effects of blending and confusion 
may well dominate measurement uncertainties. \\
For the Level 2 catalogs, the planned approach involves using the
Software Stack to deblend sources hierarchically and then
maintain this hierarchy in the catalog.
Scientifically important decisions still remain about whether
and how to use color information in the deblending, and how to divide
the flux between overlapping components. Even if the Project performs
the development work, engagement with the community can generate important
tests and figures of merit to optimize the science return.
}
~\\
\activities{
Preparations for LSST in this area involve working both with simulations
and real data. The current LSST image simulations already utilize realistic source densities,
redshift distributions, sizes, and color distributions. However, the
input galaxies do not display realistic morphologies. At least some simulations
with realistic morphologies are needed, especially for the Deep Drilling Fields. 
Inputs should come from hydrodynamical simulations,
{\it Hubble} images, and images from precursor surveys such as CFHTLS, DES and HyperSuprimeCam.
The Science Collaborations should help provide and vet inputs. \\
More challenging activities involve developing techniques and algorithms to improve the
deblending. When two galaxies at different redshifts overlap, using observations
from all the LSST filters and perhaps even EUCLID and WFIRST might 
help to disentangle them. Some attempts over the past few years have
incorporated color gradient information into the deblending algorithm, but this approach needs
much more attention for developing and testing algorithms, and 
deciding on figures-of-merit for their performance.
}
~\\
\deliverables{
Deliverables over the next several years from the activities described above include the following:
\begin{enumerate}
\item Production of realistic galaxy image inputs to the ImSim team. 
\item Development of tests and figures of merit to quantify the effects on several science objectives;
\item Assessment of the current baseline plan for Level 2 deblending and for parameter estimation for blended objects. 
\item Development of prototype implementations of deblending algorithms that take advantage of the LSST color information.
\end{enumerate}
}
\end{task}

\subsection{Optimizing Galaxy Morphology Measurements}
\tasktitle{Optimizing Galaxy Morphology Measurements}
% techniques for identifying mergers
% Bayesian techniques for inference from large data sets
% Merging human classification and machine learning
\begin{task}
\label{task:gal:morphology}
\motivation{
Morphology encodes key signatures of the formation histories of galaxies, and measurements of galaxy morphology are an important tool for constraining models of galaxy evolution. While simple measures of galaxy ellipticity and position
angles may be sufficient for the Dark Energy science goals, more sophisticated measurements of morphology are needed for galaxy-evolution science. While the ``multifit'' approach of fitting simple parametric models to galaxy profiles is useful to zeroth order, this is insufficient for the detailed morphological information that will be required for much of the galaxy science that is planned using LSST. 
\\
For well-resolved galaxies the baseline requirement is to have separate measures of bulge and disk, and spiral-arm structure, measures of concentration, asymmetry, and clumpiness. These ought to be measured as part of the level 2 processing, to enable database queries to extract subclasses of galaxies. Both parametric and non-parametric measures are desirable. While there will no doubt be optimization in level 3 processing, it is important to have enough information in the level 2 output products to pick reasonable subsets of galaxies.
\\
More importantly, while the traditional parametrisations of morphology described above will be useful, it is essential that new, more powerful methods of measuring galaxy morphology are developed and implemented, in order to leverage the exquisite volume and depth of LSST data. In this regard fast, unsupervised machine-learning techniques \citep[e.g.][]{hocking2015a}, which are able to efficiently separate LSST galaxy populations into different morphological classes, are particularly relevant and powerful (supervised techniques may not be as well-suited to LSST, since the sheer volume of data expected, coupled with the speed with which datasets will change in depth, makes producing training sets, via e.g. visual inspection, prohibitively time consuming.        
}
~\\
\activities{The preparation work will focus on defining morphological parameters and developing unsupervised machine-learning algorithms to enable users to easily query galaxy morphologies from the LSST database. Two aspects of LSST data make this a significant research project: the fact that LSST provides multi-band data with a high degree of uniformity, and the fact that the individual observations will have varying point-spread functions. The former offers the opportunity to use much more information than has been generally possible. The latter means that it will take some effort to optimize and calibrate the traditional non-parametric measures of morphology (e.g. the CAS, GINI and M20 parameters), develop new LSST-optimized parameters, and tune machine-learning algorithms to operate on this type of data. 
\\
Given the very large data set expected from LSST, which will change on short timescales in terms of depth, morphological parameters (e.g. CAS) will likely need to be calibrated on realistic data from hydro-dynamical simulations in cosmological volumes, possibly augmented by training sets classified by humans. Similarly, unsupervised machine-learning algorithms will have to be developed and implemented using a mixture of realistic simulations and precursor datasets, such as the Hyper Suprime Cam Survey. A series of ``classification challenges'' prior to the LSST survey could help to refine these techniques. 
}
~\\
\deliverables{
Deliverables over the next several years from the activities described above include the following:
\begin{enumerate}
\item Providing realistic galaxy image inputs from hydro-dynamical simulations and precursor datasets for classification tests 
\item Human classification of image subsets for calibration of morphologial parameters
\item Unsupervised machine-learning algorithms that will provide fast morphological classification of LSST datasets, developed using and implemented on precursor datasets such as the Hyper Suprime Cam Survey
\end{enumerate}
}
\end{task}


\subsection{Galaxy structural parameters}
\tasktitle{Galaxy structural parameters}

\begin{task}
\label{task:gal:struc_param}
\motivation{
The image quality provided by the camera (0.2"/pixel), the wide field coverage (9.6deg2) over 18.000 deg2 in optical and NIR (ugrizy bands), promise to provide unique data to cast light on the evolution of the internal structure of galaxies and galaxy clusters and their dark matter haloes, up to redshift well beyond unity. The full depth of the Deep-Wide-Fast (r $\sim$27.5 coadded) will allow us to identify dwarf galaxies at z~0.5, and up to $M_*+2$ galaxies in clusters and field at z~1.5, and reach enough SNR to study the structural parameters (size, Sérsic index, ellipticity etc.) of $M_*$ galaxies at z~1.2 (for seeing<0.5"). The evolution of the structural parameters, especially the size, Sersic index and B/D over billion of galaxies will allow to correlate them with mass, colours and other intrinsic properties at different epochs and clarify the mechanisms that drive the galaxy assembly and transformation. Other science directly connected is the strong lensing (in model subtracted images) and galaxy dark matter content and the dark-to-luminous mass connection in galaxies.
}
~\\
\activities{ Most of the preparatory work will consist on testing parametric methods for seeing convolved 2D fitting of the galaxy light distribution both on precursor surveys (e.g. Subaru or KiDS) and on simulated LSST images. The aim is to provide the collaboration viable tools for automatic image masking, catalogue extraction, source classification and 2D galaxy fitting. A specific aspect of this task is to optimise tool performances in order to guarantee big data analysis in reasonable time (e.g. 100 millions of LSST galaxies in all bands on a single week). The surface brightness profile of galaxies in different bands will finally generate a catalog of all relevant structural parameters via Level 3 products.
}
~\\
\deliverables{From the activities described above we expect to provide the following deliverables:
\begin{enumerate}
\item benchmarks for different existing and newly developed tools for galaxy surface photometry;
\item automatic masks (star haloes and spikes, reflections etc.) and related procedures;
\item catalog of structural parameters in different bands;
\item machine learning algorithm for the identification of faint substructures in model subtracted images (e.g. streams, merging, rings, strong lensing arcs, etc.) 
\end{enumerate}
}
\end{task}

\bigskip
\subsection{Optimizing galaxy mass profile measurement}
\tasktitle{Optimizing galaxy mass profile measurement}
% foreground lens galaxy sample selection
% completeness
% shear simulations
% galaxy-mass correlation function
% sample cuts: type, environment, color, redshift
% correlations of mass profile with optical properties  
\begin{task}
\label{task:gal:mass}
\motivation{Galaxies form and evolve dynamically via the gravitational influence
of the underlying dark matter structure.  This non-baryonic dark mass is intimately 
involved in the evolution of the baryonic component which ultimately generates the 
stellar component visible in the optical.  LSST can uniquely probe both of these
tracers for hundreds of millions of galaxies over a range of look-back time.
This provides an opportunity to probe the detailed relation between mass and light structure 
evolution for the first time. This has been attempted before in a limited way using LSST
precursor surveys.  Using 300,000 lens galaxies in the Deep Lens Survey, \citep{choi2012a} 
studied the mass profile of galaxies in three luminosity bins out to several Mpc. Using 
a similar number of lens galaxies in the COSMOS ACS data,\citep{leauthaud2012a} derived
constraints on the evolution of the stellar-to-dark matter connection in the context of halo models.
LSST will provide a billion lens galaxies and more accurate photometric redshifts .
}
~\\
\activities{
The question to be addressed is: how far down the galaxy mass function can
we expect to detect the mass profile in selected large samples of galaxies. One must start
with a model of the mass distribution in galaxies. This will
involve use of existing n-body galaxy formation simulations and resulting analytic models; 
foreground lens galaxy sample selection; the weak lens shear simulations of LSST observing over 
a large area (1000 sq.deg) containing a large sample of lens galaxies; stacked simulations of 
the galaxy-mass correlation function out to significant radii for mass environment tests (3-10 Mpc); 
sample cuts on morphological surface brightness type vs redshift; and assessment of SNR for
dwarf galaxy samples and sample completeness.  The LSST main survey will 
have hundreds of thousands of dwarf galaxies in a range of redshift from 0.2 - 0.6 which act as lenses.  
The shear SNR is high: A simulation of just 20 LSST visits to a single $z=0.5$ galaxy
with total $10^{11} M_\odot$ virial mass yields a shear SNR of 10 out to several Mpc projected radius. 
Stacking a million dwarf galaxies should thus yield high precision mass profiles, even when cut on
parameters such as mass environment, surface brightness type, stellar mass, redshift, etc.
}
~\\
\deliverables{Deliverables over the next several years from the activities 
described above include the following:
\begin{enumerate}
\item using a set of n-body full hydro models of galaxies at stages in development, 
select a set of template galaxies for use as lenses in ray-trace simulations; 
\item galaxy - mass shear simulations over at least 1000 sq.deg, using the 
latest LSST OpSim run and full end-to-end weak lens ray trace, including PSF 
and detector effects -- for a representative set of galaxies over a range of masses;
\item galaxy-mass correlation functions using these stacked LSST shear 
simulations, for sets of populations of galaxies over a wide mass range;
\item  assess SNR vs galaxy mass, and ability to correlate mass profile with 
optical surface brightness and type, over a range of redshift.  
\end{enumerate}
}
\end{task}

\subsection{Optimizing Galaxy Photometry} 
\tasktitle{Optimizing Galaxy Photometry} 
% background subtraction
% optimal co-adds
% best flux estimator
% image quality metrics 
% forced photometry with separate central point source (important for AGN)
% Using high-resolution priors where available
\begin{task}
\label{task:gal:photometry}
\motivation{
Systematic uncertainties will dominate over random uncertainties for almost any
research question one can imagine addressing with LSST. The most basic measurement
of a galaxy is its flux in each band, but this is a remarkably subtle measurement
for a variety of reasons: galaxies do not have well-defined edges, their shapes
vary, they have close neighbors, they cluster together, and lensing affects both
their brightness and clustering. These factors all affect photometry in systematic
ways, potentially creating spurious correlations that can obscure or masquerade as
astrophysical effects. For example, efforts to measure the effect of neighbors
on galaxy star-formation rates can be thrown off if the presence of a neighbor
affects the basic photometry. Measurements of galaxy magnification or measurements
of intergalactic dust can be similarly affected by systematic photometric biases.
It is thus important to hone the photometry techniques prior to the survey to 
minimize and characterize the biases. Furthermore, there are science topics that
require not just photometry for the entire galaxy, but well-characterized photometry
for sub-components, such as a central point-source or a central bulge. 
}
~\\
\activities{
The core photometry algorithms will end up being applied in level 2 processing, 
so it is important that photometry be vetted for a large number of potential
science projects before finalizing the software. Issues include the following.
(1) Background estimation, which, for example, can greatly affect the photometry
for galaxies in clusters or dwarfs around giant galaxies. (2) Quantifying the
biases of different flux estimators vs. (for example) distances to and fluxes
of their neighbors. (3) Defining optimal strategies to deal with the varying
image quality. (4) Defining a strategy for forced photometry of a central point
source. For time-varying point-sources, the image subtractions will give a
precise center, but will only measure the AC component of the flux. Additional
measurements will be needed to give the static component. (5) Making use
of high-resolution priors from either Euclid or WFIRST, when available. 
Because photometry is so central to much of LSST science, there will need to
be close collaboration between the LSST Project and the community. 
}
~\\
\deliverables{
Deliverables over the next several years from the activities described above include the following:
\begin{enumerate}
\item developing metrics for various science cases to help evaluate the level 2 photometry;
\item providing realistic inputs for difficult photometry cases (e.g. close neighbors, clusters of galaxies, AGN in galaxies of various morphologies);
\end{enumerate}
Deliverables over the longer term include developing optimal techniques for forced photometry
using priors from the space missions.
}
\end{task}

\subsection{Optimizing Measurements of Stellar Population Parameters} 
\tasktitle{Optimizing Measurements of Stellar Population Parameters} 
% Strategies for dealing with strong covariance of parameter estmates
\begin{task}
\label{task:gal:stellarpops}
\motivation{
The colors of galaxies carry information about their star-formation histories, 
each interval of redshift being a snapshot of star-formation up until that time.
Unfortunately, estimates of star-formation rates and star-formation histories 
for a single galaxy based on only the LSST bands will be highly uncertain, 
due largely to degeneracies between age, dust extinction and metallicity. 
Strategies for overcoming the degeneracies include hierarchical modeling -- using
ensembles of galaxies to constrain the hyper-parameters that govern 
the star-formation histories of sets of galaxies rather than individuals, 
and using ancillary data from other wavelengths. 
}
~\\
\activities{
Activities in this area include developing scalable techniques for 
hierarchical Bayesian inference on very large data sets. These can be
tested on semi-analytical or hydrodynamical models, where the answer is known,
even if it does not correctly represent galaxy evolution. The models should
also be analyzed to find simple analytical expressions for star-formation
histories, chemical evolution and the evolution and behavior of dust to
make the Bayesian inference practical.\\
Another important activity is to identify the ancillary data sets and
observing opportunities, especially for the deep fields.
}
~\\
\deliverables{
Deliverables over the next several years from the activities described above include the following:
\begin{enumerate}
\item developing and refining techniques for constraining star-formation histories of large ensembles of galaxies;
\item providing model inputs to guide in developing these techniques;
\item refining the science requirements for ancillary multi-wavelength data to support LSST.
\end{enumerate}
}
\end{task}


%\begin{comment}

%\subsection{Software Integration}
%\tasktitle{Software Integration}
% Level 2 and Level 3 software
%\begin{task}
%\label{task:gal:integration}
%\motivation{
%The LSST Project is responsible for level 2 data processing, and the community
%is expected to any processing beyond that as level 3. Furthermore, some algorithms developed
%as part of the level 3 effort are expected to migrate to level 2. There needs
%to be strong coordination between the Project and the community for this concept
%to work. This includes training in developing level 3 software and community engagement in 
%defining the requirements and interfaces. 
%}
%~\\
%\activities{
%The most urgent activity is to develop some early prototypes of level 3 software
%so that the interfaces can be worked out on realistic use cases.
%}
%~\\
%\deliverables{ADD SOMETHING}
%\end{task}

%\end{tasklist}
%\end{comment}

\end{tasklist}
}
