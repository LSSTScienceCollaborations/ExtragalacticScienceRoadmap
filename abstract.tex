\begin{center}

\vspace*{30mm}

{\bf Abstract.} 

This white paper describes the LSST Dark Energy Science Collaboration (DESC), whose goal is the study of dark energy and related topics in fundamental physics with data from the Large Synoptic Survey Telescope (LSST).  
It provides an overview of dark energy science and describes the current and anticipated state of the field.  
It makes the case for the DESC by laying out a robust analytical framework for dark energy science that has been defined by its members and the comprehensive three-year work plan they have developed for implementing that framework.  
The analysis working groups cover five key probes of dark energy: weak lensing, large scale structure, galaxy clusters, Type Ia supernovae, and strong lensing. 
The computing working groups span cosmological simulations, galaxy catalogs, photon simulations and a systematic software and computational framework for LSST dark energy data analysis. 
The technical working groups make the connection between dark energy science and the LSST system. 
The working groups have close linkages, especially through the use of the photon simulations to study the impact of instrument design and survey strategy on analysis methodology and cosmological parameter estimation. 
The white paper describes several high priority tasks identified by each of the 16 working groups. 
Over the next three years these tasks will help prepare for LSST analysis, make synergistic connections with ongoing cosmological surveys and provide the dark energy community with state of the art analysis tools. 
Members of the community are invited to join the DESC, according to the membership policies described in the white paper.
Applications to sign up for associate membership may be made by submitting the Web form at
\url{http://www.slac.stanford.edu/exp/lsst/desc/signup.html}
with a short statement of the work they wish to pursue that is relevant to the DESC.


\end{center}
